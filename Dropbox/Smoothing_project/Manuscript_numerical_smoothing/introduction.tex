Many option pricing problems  require the computation of multivariate integrals. The dimension of these integrals
is determined by the number of independent stochastic factors (e.g. the number of time steps in the time discretization or the number of assets under consideration).  The high dimension of these integrals can be treated with dimension-adaptive quadrature methods to have the desired convergence behavior.

Unfortunately, in many cases, the  integrand has lower regularity, and contains either kinks and jumps. In fact, an option is normally considered worthless if the value falls below a predetermined strike price.  A kink  (discontinuity in the gradients) is present when the payoff function is  continuous, while a jump (discontinuity in the function)  exists when the payoff corresponds to a binary or other digital options. The existence of kinks or jumps in the integrand  heavily degrades
the performance of quadrature formulas.  In this work, we are interested in solving this problem  by using adaptive  sparse grids quadrature (ASGQ) methods coupled with suitable transformations. The main idea is to find lines or areas of discontinuity and to employ suitable transformations of the integration domain. Then  by a pre-integration (smoothing) step with respect to the dimension containing the kink/jump,  we end up with integrating  only over the smooth parts of the integrand and the fast convergence of the sparse grid method can be regained.

One can ignore the kinks and jumps, and apply directly a method for integration over $\rset^d$.  Despite the  significant progress in SG methods \cite{bungartz2004sparse} for high dimensional integration  of  smooth integrands, few works have been done to deal with  cases involving integrands with kinks or jumps due to the decreasing performance of SG methods in the presence of kinks and jumps. 

Some works \cite{griebel2013smoothing,bayersmoothing, griebel2017note,griewank2017high,xiao2018conditional} addressed similar kind of problems, characterized by the presence of kinks and jumps,  but with much more emphasis on Quasi Monte Carlo (QMC). In \cite{griebel2013smoothing, griebel2017note,griewank2017high}, an  analysis of the performance of  Quasi Monte Carlo (QMC) and SG methods has been conducted, in the presence of kinks and jumps.  In \cite{griebel2013smoothing,griebel2017note}, the authors studied the terms of the ANOVA decomposition of functions with kinks defined on $d$-dimensional Euclidean space $\rset^d$, and showed   that under some assumptions all but the the highest order ANOVA term  of the $2^d$ ANOVA terms can be smooth for the case of an arithmetic Asian option with the Brownian bridge construction. Furthermore, \cite{griewank2017high} extended the work in \cite{griebel2013smoothing,griebel2017note} from kinks
to jumps for  the case of an arithmetic average digital Asian option with the principal component analysis (PCA). The main findings in \cite{griebel2013smoothing,griebel2017note} was obtained  for an  integrand  of the form $f(\mathbf{x}) = \max(\phi(\mathbf{x}), 0)$ with $\phi$ being smooth. In fact, by assuming  i) the $d$-dimensional function $\phi$ has a positive partial derivative with respect to $x_j$ for some $j \in \{1,\dots,d\}$, ii) certain growth conditions at infinity are satisfied, the authors showed that the ANOVA terms of $f$ that do not depend on the variable $x_j$ are smooth.   We note that \cite{griebel2013smoothing,griebel2017note,griewank2017high} focus  more on  theoretical aspects of applying QMC in such a setting. On the other hand, we focus more on  specific practical problems, where we add the adaptivity paradigm to the picture.

A recent work \cite{xiao2018conditional} addresses similar kind of problems using QMC. Being very much related to \cite{bayersmoothing}, the authors i) assume that the conditional expectation can be computed explicitly, by imposing very strong assumptions. ii) Secondly, they  use  PCA on the gradients to reduce the effective dimension. In our work, we do not make such strong assumptions, which is why we need numerical methods, more precisely root finding and the quadrature in the first direction.

In the first part of this work, we design a novel efficient pricing method based on i) numerical smoothing based on root finding procedure, and ii) solving the integration problem using hierarchical ASGQ coupled with Brownian bridges and Richardson
Extrapolation. The main contribution of this work is showing  the potential and advantage of the numerical smoothing idea when using deterministic quadrature (by providing more regularity structure implying better convergence behavior for ASGQ) and when using MLMC (by playing the role of variance reduction that even improves the rate of complexity of the MLMC). 

Usually, for such  tasks, standard MLMC will fail or not have the optimal performance, due to the singularity present in the delta or the indicator functions, implying either high variance or kurtosis for the MLMC estimator. In the literature, few works tried to address this issue. For instance,
\begin{enumerate}
\item Avikainen in \cite{avikainen2009irregular}, and Giles, Higham, and Mao in \cite{giles2009analysing} used MLMC for such a task without smoothing.
\item On the other hand, a second approach  was suggested in  \cite{giles2008improved,giles2013numerical}, that used implicit smoothing based on the use of conditional expectations. There are two potential issues  with this second approach: i) In general cases, one may have dynamics where it is not easy to derive an  analytic expression for the conditional expectation and ii) This approach used a higher order scheme, that is the Milstein scheme, to improve the strong order of convergence, and consequently the complexity of the MLMC estimator. Such a scheme becomes very computationally expensive for higher dimensional dynamics.
\item In \cite{giles2015multilevel}, the authors suggested a different approach based on parametric smoothing.  In fact, they carefully constructed a regularized version of the QoI, based on a 
regularization parameter that depends on the weak and strong convergence rates and also  the tolerance requirement.  This approach, despite offering better performance for the MLMC estimator and a better setting for theoretical analysis, it has the practical disadvantage consisting in the difficulty of generalizing it to cases where there is no prior knowledge of the the convergence rates (that is they need to be estimated numerically), and also for each error tolerance, a new  regularization parameter needs to be computed.
\end{enumerate}    

In this work, we address a similar problem and  propose an alternative approach that is  based on numerical smoothing. Our approach compared to previous mentioned works, has the following advantages
\begin{itemize}
\item It can be easily applied to cases where one can not apply analytic smoothing.
\item We obtain better rates of strong convergence than those obtained using Euler scheme without smoothing as in \cite{giles2015multilevel}, and  similar  rates of strong convergence and MLMC complexity  as in  \cite{giles2008improved,giles2013numerical}, without the need to use higher order schemes such as Milstein scheme. 
\item Our approach is parameter free compared to that of  \cite{giles2015multilevel}. Therefore, in practice it is much easier to apply for any dynamics and QoI.
\end{itemize} 

