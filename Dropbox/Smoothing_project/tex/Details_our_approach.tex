In the following, we describe our approach which can be seen as a two stage method. In the first stage, we use root finding procedure to perform the numerical smoothing described in Section \ref{sec:Step $1$: Numerical smoothing}, then in a second stage we perform the numerical integration to compute \eqref{eq: pre_integration_step_wrt_y1_basket}, described in Section \ref{sec:Step $2$: Integration}, by employing hierarchical adaptive sparse grids  quadrature, using the same construction as in MISC, proposed in \cite{haji2016multi}. Therefore, the initial integration problem that we are solving lives in $dN-1$-dimensional space, which becomes very large as either the number of time steps $N$, used in the discretization  scheme, increases, or the number the assets increase. 

We describe the  MISC method in our context in Section \ref{sec:Details of the MISC}.  To make an effective use of MISC, we  apply two transformations to overcome the issue of facing a high dimensional integrand. The first transformation consists of applying a hierarchical  path generation method, based on Brownian bridge (Bb) construction, with the aim of reducing the effective dimension as  described  in Section \ref{sec:Brwonian bridge construction}. The second transformation consists of applying Richardson extrapolation to reduce the bias, resulting in reducing  the maximum number of dimensions needed for the integration problem. Details about  Richardson extrapolation  are provided in Section \ref{sec:Richardson extrapolation}.

Since $g$ can have a kink  or jump. Computing $h$ in \eqref{eq:smooth_function_after_pre_integration}  should be carried carefully to not deteriorate the smoothness of $h$. This can be done by applying a root finding procedure and then computing the uni-variate integral by summing the terms coming from integrating in each region where $g$ is smooth. We provide details about the root finding procedure in Section \ref{sec: Root Finding}.

If we denote by $\mathcal{E}_{\text{tot}}$ the total error of approximating the  expectation in \eqref{eq: pre_integration_step_wrt_y1_basket} using the MISC estimator, $Q_N$, then we have a natural error decomposition
\begin{align}\label{eq:total_error}
\mathcal{E}_{\text{tot}} & \le \abs{\expt{g(\mathbf{X}(T))}- \expt{h(\mathbf{y}_{-1}, \mathbf{z}^{(1)}_{-1},\dots,\mathbf{z}^{(d)}_{-1} )}}+\abs{\expt{h(\mathbf{y}_{-1}, \mathbf{z}^{(1)}_{-1},\dots,\mathbf{z}^{(d)}_{-1} )}-Q_{N}}\nonumber\\
  & \le \mathcal{E}_B(N)+ \mathcal{E}_Q(\text{TOL}_{\text{MISC}},N),
\end{align}
where  $\mathcal{E}_Q$ is the quadrature error, $\mathcal{E}_B$  is the bias.

\subsection{Adaptive Sparse Grids}\label{sec:Details of the MISC}

We assume that we want to approximate the expected value $\text{E}[f(Y)]$ of an analytic function $f\colon \Gamma \to \rset$ using a tensorization of quadrature formulas over $\Gamma$.

To introduce simplified notations, we start with the one-dimensional case. Let us denote by $\beta$ a non negative integer, referred to as a ``stochastic discretization level", and by $m: \nset \rightarrow \nset$  a strictly increasing function with $m(0)=0$ and $m(1)=1$, that we call  ``level-to-nodes function". At level $\beta$, we consider a set of $m(\beta)$ distinct quadrature points in $\rset$, $\mathcal{H}^{m(\beta)}=\{y^1_\beta,y^2_\beta,\dots,y_\beta^{m(\beta)}\} \subset \rset$, and a set of quadrature weights, $\boldsymbol{\omega}^{m(\beta)}=\{\omega^1_\beta,\omega^2_\beta,\dots,\omega_\beta^{m(\beta)}\}$. We also let $C^0(\rset)$ be the set of real-valued continuous functions over $\rset$. We then define the quadrature operator as
\begin{equation*}
Q^{m(\beta)}:C^0(\rset) \rightarrow \rset, \quad Q^{m(\beta)}[f]= \sum_{j=1}^{m(\beta)} f(y^j_\beta) \omega_\beta^j.
\end{equation*}
In our case, we have in \eqref{eq: pre_integration_step_wrt_y1_basket} a multi-variate integration problem with, $f:=h$, $\mathbf{Y}=(\mathbf{y}_{-1},\mathbf{z}^{(1)}_{-1},\dots,\mathbf{z}^{(d)}_{-1})$,  and  $\Gamma=\rset^{dN-1}$, in the previous notations.  Furthermore, since we are dealing with Gaussian densities, using Gauss-Hermite quadrature points is the appropriate choice.

We define for any multi-index $\boldsymbol{\beta} \in \nset^{dN-1}$
$$Q^{m(\boldsymbol{\beta})}: C^0(\rset^{dN-1}) \rightarrow \rset,\quad  Q^{m(\boldsymbol{\beta})}= \bigotimes_{n = 1}^{dN-1} Q^{m(\beta_n)} \COMMA $$
where the $n$-th quadrature operator is understood to act only on the $n$-th variable of $f$. Practically, we obtain the value of $Q^{m(\boldsymbol{\beta})}[f]$  by considering the tensor grid $\mathcal{T}^{m(\boldsymbol{\beta})}= \times_{n = 1}^{dN-1}  \mathcal{H}^{m(\beta_n)}$ with cardinality $\#\mathcal{T}^{m(\boldsymbol{\beta})}=\prod_{n=1}^{dN-1} m (\beta_n)$ and computing

$$ Q^{m(\boldsymbol{\beta})}[f]= \sum_{j=1}^{\#\mathcal{T}^{m(\boldsymbol{\beta})}} f(\hat{y}_j) \bar{\omega}_j \COMMA$$
where $\hat{y}_j \in \mathcal{T}^{m(\boldsymbol{\beta})}$ and $\bar{\omega}_j$ are  products of weights of the univariate quadrature rules.

A direct approximation $\expt{f[\mathbf{Y}]} \approx Q^{\boldsymbol{\beta}}[f]$ is not an appropriate option  due to the well-known ``curse of dimensionality". We use  a hierarchical adaptive sparse grids\footnote{More details about sparse grids can be found in \cite{bungartz2004sparse}.} quadrature strategy, specifically using the same
construction as MISC, and which uses  stochastic discretizations  and a classic sparsification approach to obtain an effective approximation scheme for $\expt{f}$. 

To be concrete, in our setting, we are left with a $dN-1$-dimensional Gaussian random input, which is chosen independently, resulting in  $dN-1$ numerical parameters for MISC, which we use as the basis of the multi-index construction. For a multi-index $\boldsymbol{\beta} = (\beta_n)_{n=1}^{dN-1} \in \mathbb{N}^{dN-1}$, we denote  by
$Q_N^{\boldsymbol{\beta}}$,   the result of approximating \eqref{eq: pre_integration_step_wrt_y1_basket} with a number of quadrature points  in the $i$-th dimension equal to  $m(\beta_i)$. We further define the set of
differences $\Delta Q_N^{\boldsymbol{\beta}}$ as follows: for a single index $1 \le i \le dN-1$,
let
\begin{equation*}
\Delta_i Q_N^{\boldsymbol{\beta}} = \left\{ 
\aligned 
 Q_N^{\boldsymbol{\beta}} &- Q_N^{\boldsymbol{\beta}'}  \text{, with } \boldsymbol{\beta}' =\boldsymbol{\beta} - e_i, \text{ if } \boldsymbol{\beta}_i>0 \COMMA \\
 Q_N^{\boldsymbol{\beta}} &, \quad  \text{ otherwise,}
\endaligned
\right.
\end{equation*}
where $e_i$ denotes the $i$th $dN-1$-dimensional unit vector. Then, $\Delta
Q_N^{\boldsymbol{\beta}}$ is defined as
\begin{equation*}
\Delta Q_N^{\boldsymbol{\beta}} = \left( \prod_{i=1}^{dN-1} \Delta_i \right) Q_N^{\boldsymbol{\beta}}.
\end{equation*}
The MISC estimator used for approximating \eqref{eq: pre_integration_step_wrt_y1_basket}, and using a set of multi-indices $\mathcal{I}\subset \nset^{dN-1}$ is given by
\begin{equation}\label{eq:MISC_quad_estimator}
	Q_N^{\mathcal{I}} = \sum_{\boldsymbol{\beta} \in \mathcal{I}} \Delta Q_N^{\boldsymbol{\beta}}.
\end{equation}
The quadrature error in this  case  is given by
\begin{equation}\label{eq:quadrature error}
\mathcal{E}_Q(\text{TOL}_{\text{MISC}},N) =\abs{Q_N^\infty - Q_N^\mathcal{I}} \le \sum_{\ell \in \mathbb{N}^{dN-1} \setminus
	\mathcal{I}} \abs{\Delta Q_N^\ell}.
\end{equation}
We define the work contribution, $\Delta \mathcal{W}_{\boldsymbol{\beta}}$, to be the computational cost  required to add  $\Delta Q_N^{\boldsymbol{\beta}}$ to $Q^{\mathcal{I}}_N$, and the error contribution, $\Delta E_{\boldsymbol{\beta}}$, to be  a measure of how much the quadrature error, defined in \eqref{eq:quadrature error}, would decrease once $\Delta Q_N^{\boldsymbol{\beta}}$  has been added to  $Q^{\mathcal{I}}_N$, that is 
\begin{align}\label{eq:Work_error_contributions}
\Delta \mathcal{W}_{\boldsymbol{\beta}} &= \text{Work}[Q^{\mathcal{I} \cup \{\boldsymbol{\beta}\}}_N]-\text{Work}[Q^{\mathcal{I}}_N] \nonumber\\
\Delta E_{\boldsymbol{\beta}} &= \abs{Q^{\mathcal{I} \cup \{\boldsymbol{\beta}\}}_N-Q^{\mathcal{I}}_N}.
\end{align}
 The  construction of the optimal  $\mathcal{I}$ will be done by profit thresholding, that is, for a certain threshold value $\bar{T}$, and a profit of a hierarchical surplus defined by
 \begin{equation*}
 P_{\boldsymbol{\beta}}= \frac{\abs{\Delta E_{\boldsymbol{\beta}}}}{\Delta\mathcal{W}_{\boldsymbol{\beta}}},
 \end{equation*}
  the optimal index set  $\mathcal{I}$  for MISC  is given by 
 $\mathcal{I}=\{\boldsymbol{\beta}: P_{\boldsymbol{\beta}}	 \ge \bar{T}\}$.
 
\begin{remark}
The analiticity assumption, stated in the beginning of Section \ref{sec:Details of the MISC}, is crucial for the optimal performance of our proposed method. In fact, although we face the issue of the  ``curse of dimensionality" when increasing $N$, the analiticity of $f$ implies a spectral convergence for sparse grids quadrature. A discussion about the analiticity of our integrand is provided in Section \ref{sec:Analiticity Analysis}.
\end{remark}  
 
\subsection{Brownian bridge (Bb) construction}\label{sec:Brwonian bridge construction}
In the literature of adaptive sparse grids and  QMC, several hierarchical path generation methods (PGMs) or transformation methods have been proposed to reduce the effective dimension. Among these transformations, we cite  the Brownian bridge (Bb)  construction \cite{morokoff1994quasi,moskowitz1996smoothness,caflisch1997valuation}, the principal component analysis (PCA)  \cite{acworth1998comparison} and the linear transformation (LT) \cite{imai2004minimizing}.

In our context, the Brownian motion on a time discretization  can be constructed either sequentially using a standard random walk construction, or hierarchically using   other PGMs as listed above. For our purposes, to make an effective use of MISC, which benefits from anisotropy, we use the Bb construction since it produces  dimensions with different importance for MISC, contrary to a random walk procedure for which all the dimensions of the stochastic space have equal importance. In fact, Bb uses the first several coordinates of the low-discrepancy points to determine the general shape of the Brownian path, and the last few coordinates influence only the fine detail of the path. Consequently, this transformation  reduces the effective dimension  of the problem, which results in accelerating the MISC method by reducing the computational cost.

Let us denote $\{t_i\}_{i=0}^{N}$ the grid of time steps. Then the Bb construction \cite{glasserman2004monte} consists of the following: given a past value $B_{t_i}$ and a future value $B_{t_k}$, the value $B_{t_j}$ (with $t_i < t_j < t_k$) can be generated according to 
\begin{equation*}
B_{t_j}=(1-\rho) B_{t_i}+\rho B_{t_k}+ \sqrt{\rho (1-\rho)(k-i) \Delta t} z, \: z \sim \mathcal{N}(0,1) \COMMA
\end{equation*}
where $\rho=\frac{j-i}{k-i}$.  

\subsection{Richardson extrapolation}\label{sec:Richardson extrapolation}
Another transformation that we couple with MISC is Richardson extrapolation \cite{talay1990expansion}. In fact, applying level $K_\text{R}$ (level of extrapolation) of Richardson extrapolation  dramatically reduces the bias, and as a consequence reduces the  number of time steps $N$ needed in the coarsest level to achieve a certain error tolerance. As a consequence, Richardson extrapolation directly reduces  the total dimension of the integration problem for achieving some error tolerance.

Let us denote by $(X_t)_{0 \le t \le T}$ a certain stochastic process and by $(\hat{X}_{t_i}^h)_{0 \le  t_i \le T}$ its approximation using a suitable  scheme with a time step $h$.  Then, for sufficiently small $h$, and a suitable smooth function $f$, we assume that
\begin{align}\label{Euler_weak_error_strenghten}
	\expt{f(\hat{X}_T^h)}= \expt{f(X_T)} + c h +\Ordo{h^2} \PERIOD
\end{align}
Applying \eqref{Euler_weak_error_strenghten} with discretization step $2h$, we  obtain
\begin{align*}
	\expt{f(\hat{X}_T^{2h})}= \expt{f(X_T)} + 2 c h +\Ordo{h^2} \COMMA
\end{align*}
implying
\begin{align*}
	2 \expt{f(\hat{X}_T^{2h})}- \expt{f(\hat{X}_T^{h})} =\expt{f(X_T)} + \Ordo{h^2} \PERIOD
\end{align*}
For higher levels of extrapolations, we use the following: Let us denote by $h_J=h_0 2^{-J}$ the grid sizes (where $h_0$ is the coarsest grid size), by $K_\text{R}$ the level of the Richardson extrapolation, and by $I(J,K_\text{R})$ the approximation of $\expt{f((X_T)}$ by terms up to level $K_\text{R}$ (leading to a weak error of order $K_\text{R}$), then we have the following recursion 
\begin{align*}
I(J,K_\text{R})=\frac{2^{K_\text{R}}\left[I(J,K_\text{R}-1)-I(J-1,K_\text{R}-1)\right]}{2^{K_\text{R}}-1},\quad J=1,2,\dots, K_\text{R}=1,2,\dots
\end{align*}

\subsection{Root Finding}\label{sec: Root Finding}
From Section \ref{sec:Discrete time, practical motivation}, we denoted the location of irregularity (the kink) by $y_1^\ast$, that is $G$, defined in \eqref{eq: pre_integration_step_wrt_y1_basket}, is not smooth at the point $(y_1^\ast, \mathbf{y}_{-1},\mathbf{z}^{(1)}_{-1},\dots,\mathbf{z}^{(d)}_{-1})$. Let us call $R$ the mapping such that: $R: (\mathbf{y}_{-1},\mathbf{z}^{(1)}_{-1},\dots,\mathbf{z}^{(d)}_{-1} ) \rightarrow y_1^\ast$. Generally, there might be, for given $ (\mathbf{y}_{-1},\mathbf{z}^{(1)}_{-1},\dots,\mathbf{z}^{(d)}_{-1} )$
\begin{itemize}
	\item no solution, i.e., the integrand in the definition of $h( \mathbf{y}_{-1},\mathbf{z}^{(1)}_{-1},\dots,\mathbf{z}^{(d)}_{-1} )$ above
	is smooth (\textit{best case});
	\item a unique solution;
	\item multiple solutions.
\end{itemize}
Generally, we need to assume that we are in the first or second
case. Specifically, we need that
\begin{equation*}
	( \mathbf{y}_{-1},\mathbf{z}^{(1)}_{-1},\dots,\mathbf{z}^{(d)}_{-1} ) \mapsto h( \mathbf{y}_{-1},\mathbf{z}^{(1)}_{-1},\dots,\mathbf{z}^{(d)}_{-1} ) \text{ and } ( \mathbf{y}_{-1},\mathbf{z}^{(1)}_{-1},\dots,\mathbf{z}^{(d)}_{-1} ) \mapsto \hat{h}( \mathbf{y}_{-1},\mathbf{z}^{(1)}_{-1},\dots,\mathbf{z}^{(d)}_{-1} )
\end{equation*}
are smooth, where $\hat{h}$ denotes the numerical approximation of $h$ based
on a grid containing $R( \mathbf{y}_{-1},\mathbf{z}^{(1)}_{-1},\dots,\mathbf{z}^{(d)}_{-1} )$. In particular, $R$ itself should be smooth
in $( \mathbf{y}_{-1},\mathbf{z}^{(1)}_{-1},\dots,\mathbf{z}^{(d)}_{-1} )$. This would already be challenging in practice in the third case. Moreover, in the general situation we expect the number of solutions to increase when the discretization of the SDE gets finer.

In many situations, case $2$ (which is thought to include case 1) can be
guaranteed by monotonicity (see assumption  \eqref{assump:Monotonicity condition}). For instance, in the case of one-dimensional SDEs
with $z_1$ representing the terminal value of the underlying Brownian motion, this can often be seen from the SDE itself. Specifically, if each increment ``$dX$'' is increasing in $z_1$, no matter the value of $X$, then the solution $X_T$ must be increasing
in $z_1$. This is easily seen to be true in examples such as the Black-Scholes model and the CIR process. (Strictly speaking, we have to distinguish between the continuous and discrete time solutions. In these examples, it does not matter.) On the other hand, it is also quite simple to construct counter examples, where monotonicity fails, for instance SDEs for which the ``volatility'' changes sign, such as a trigonometric function.\footnote{Actually, in every such case the simple remedy is to replace the volatility by its absolute value, which does not change the law of the solution. Hence, there does not seem to be a one-dimensional counter-example.}

Even in multi-dimensional settings, such monotonicity conditions can hold in specific situations. For instance, in case of a basket option in a multivariate Black Scholes framework, we can choose a linear combination of the terminal values of the driving Bm, denoted by $Y_1$ in Section \ref{sec:Step $1$: Numerical smoothing}, such that the basket is a monotone function of $y_1$. (The coefficients of the linear combination will depend on the correlations and the weights of the basket.) However, in that case this may actually not correspond to the optimal ``rotation'' in terms of optimizing the smoothing effect.

