%%%%%%%%%%%%
% commands %
%%%%%%%%%%%%

%packages
\usepackage[leqno]{amsmath}
\usepackage{amsthm}
\usepackage{bbm}
\usepackage{enumerate}
\usepackage[round,longnamesfirst]{natbib}
\usepackage[bookmarksopen,pdfstartview=FitH]{hyperref}
\hypersetup{colorlinks,%
            citecolor=blue,%
            filecolor=blue,%
            linkcolor=blue,%
            urlcolor=blue}%
\usepackage{dsfont}
\usepackage{mathtools}
\usepackage[mathscr]{euscript}
\usepackage{nicefrac,booktabs,mathrsfs}
\usepackage{nth}
\usepackage[dvipsnames]{xcolor}
%\usepackage[color]{showkeys}
\definecolor{refkey}{gray}{0.5}
\definecolor{labelkey}{gray}{.7}
\usepackage[backgroundcolor=white,bordercolor=orange]{todonotes}
\usepackage{upgreek}
\usepackage[dvipsnames]{xcolor}
%%%%%%%%%%%%%%%%%%%%%%%%%%%%%%%%%%%%%%
\usepackage{amssymb}
\usepackage[mathscr]{euscript}
\usepackage{xspace}
\usepackage{epsfig,rotating}
\usepackage{graphicx}
\input{xy}\xyoption{all}

\usepackage{lscape}
\usepackage{enumitem}
\usepackage{upgreek}
\usepackage{subfigure} 
\usepackage{epstopdf}

% \usepackage{txgreeks}
% \usepackage[T1]{fontenc}

\usepackage[left=1.10in,right=1.10in,top=1.40in,bottom=1.30in]{geometry}

%%%%%%%%%%%%%%%%%%%%%%%%%%%%%%%%%%%%%%%%%%%%%%%

\DeclareMathAlphabet{\mathpzc}{OT1}{pzc}{m}{it}


%%%%%%%%%%%%%

%Textstyles


%letters 
\renewcommand{\P}{\mathbb{P}} %probability
\newcommand{\E}{\mathbb{E}} %expectation
\newcommand{\FF}{\mathbb{F}}
\newcommand{\R}{\mathbb{R}}
\newcommand{\cF}{\mathcal{F}}
\newcommand{\cP}{\mathcal{P}}
\newcommand{\cD}{\mathcal{D}}


%common math symbols
\def\half{\frac1{2}}
\newcommand{\ind}{\mathbbm1}

%other symbold, common names
\newcommand{\cadlag}{c{\'a}dl{\'a}g }
\newcommand{\Levy}{L{\'e}vy }


%common math notation
\newcommand{\cond}[1]{\left.#1\right\vert}%to get a vert with the right size

%brackets
\newcommand{\brac}[1]{\ensuremath{\left(#1\right)}}
\newcommand{\Brac}[1]{\ensuremath{\big(#1\big)}}
\newcommand{\BRac}[1]{\ensuremath{\Big(#1\Big)}}
\newcommand{\BRAc}[1]{\ensuremath{\bigg(#1\bigg)}}
\newcommand{\BRAC}[1]{\ensuremath{\Bigg(#1\Bigg)}}

\newcommand{\brak}[1]{\ensuremath{\left[#1\right]}}
\newcommand{\scal}[1]{\ensuremath{\left\langle#1\right\rangle}}
\newcommand{\set}[1]{\ensuremath{\left\{#1\right\}}}
\newcommand{\Norm}[1]{\left\Vert#1\right\Vert}%doudle vert norm
\newcommand{\norm}[1]{\left\vert#1\right\vert}%single vert norm




%%%%%%%%%%%%%%%%%%%%%%%%%%%%%%%%%%%%%%%%


%% ========================================================================== %%
%% ========================================================================== %%

\theoremstyle{plain}
\newtheorem{theorem}{Theorem}[section]
\newtheorem{lemma}[theorem]{Lemma}
\newtheorem{proposition}[theorem]{Proposition}
\newtheorem*{proposition*}{Proposition}
\newtheorem{claim}[theorem]{Claim}
\newtheorem{corollary}[theorem]{Corollary}
\newtheorem{axiom}{Axiom}
\newtheorem{hypothesis}[theorem]{Hypothesis}  %[section]



\theoremstyle{definition*}
\newtheorem{example*}{Example}
\theoremstyle{definition}
\newtheorem{remark}[theorem]{Remark}
\newtheorem{note}{Note}[section]
\newtheorem{definition}[theorem]{Definition}
\newtheorem{example}[theorem]{Example}
\newtheorem*{ackn}{Acknowledgements}
\newtheorem{assumption}{Assumption}
\newtheorem{approach}{Approach}
\newtheorem{critique}{Critique}
\newtheorem{question}{Question}
\newtheorem{aim}{Aim}
\newtheorem*{assucd}{Assumption ($\mathbb{CD}$)}
\newtheorem*{asa}{Assumption ($\mathbb{A}$)}
\newtheorem*{appS}{Approximation ($\mathbb{S}$)}
\newtheorem*{appBS}{Approximation ($\mathbb{BS}$)}
% \newtheorem*{asp}{Assumption ($\P$)}
% \newtheorem*{ass}{Assumption ($\mathbb{S}$)}
%% 


%% ========================================================================== %%
\renewcommand{\theequation}{\thesection.\arabic{equation}}
\numberwithin{equation}{section}

\renewcommand{\thefigure}{\thesection.\arabic{figure}}
\numberwithin{equation}{section}

\newcommand{\Law}{\ensuremath{\mathop{\mathrm{Law}}}}
\newcommand{\loc}{{\mathrm{loc}}}

\let\SETMINUS\setminus
\renewcommand{\setminus}{\backslash}

\def\stackrelboth#1#2#3{\mathrel{\mathop{#2}\limits^{#1}_{#3}}}

% \makeatletter
% \def\Ddots{\mathinner{\mkern1mu\raise\p@
% \vbox{\kern7\p@\hbox{.}}\mkern2mu
% \raise4\p@\hbox{.}\mkern2mu\raise7\p@\hbox{.}\mkern1mu}}
% \makeatother
% 
% \newcommand\llambda{{\mathchoice
%      {\lambda\mkern-4.5mu{\raisebox{.4ex}{\scriptsize$\backslash$}}}
%      {\lambda\mkern-4.83mu{\raisebox{.4ex}{\scriptsize$\backslash$}}}
%      {\lambda\mkern-4.5mu{\raisebox{.2ex}
% {\footnotesize$\scriptscriptstyle\backslash$}}}
%      {\lambda\mkern-5.0mu{\raisebox{.2ex}
% {\tiny$\scriptscriptstyle\backslash$}}}}}

\newcommand{\prozess}[1][L]{{\ensuremath{#1=(#1_t)_{t\in[0,T]}}}\xspace}
\newcommand{\prazess}[1][L]{{\ensuremath{#1=(#1_t)_{t\ge0}}}\xspace}
\newcommand{\pt}[1][N]{\ensuremath{\P_{#1}}\xspace}
\newcommand{\tk}[1][N]{\ensuremath{T_{#1}}\xspace}
\newcommand{\dd}[1][]{\ensuremath{\ud{#1}}\xspace}


\newcommand{\bscal}[2]{\ensuremath{\big\langle #1, #2 \big\rangle}}

%\newcommand{\R}[1][\R]{\ensuremath{\R^{#1}}\xspace}
% \newcommand{\pt}[1][]{{\ensuremath{\P_{T^*_{#1}}}}\xspace}
% \newcommand{\ts}[1][]{\ensuremath{T^*_{#1}}\xspace}
%% ========================================================================== %%
%% ========================================================================== %%
\def\lev{L\'{e}vy\xspace}
\def\lk{L\'{e}vy--Khintchine\xspace}
\def\lib{LIBOR\xspace}
\def\mg{martingale\xspace}
\def\smmg{semimartingale\xspace}
\def\alm{affine LIBOR model\xspace}
\def\alms{affine LIBOR models\xspace}
\def\dalms{defaultable \alms}
\def\ap{affine process\xspace}
\def\aps{affine processes\xspace}

\def\half{\frac1{2}}

\def\F{\ensuremath{\mathcal{F}}}
\def\bD{\mathbf{D}}
\def\bF{\mathbf{F}}
\def\bG{\mathbf{G}}
\def\bH{\mathbf{H}}
\def\R{\ensuremath{\mathbb{R}}}
\def\Rp{\mathbb{R}_{\geqslant0}}
\def\Rm{\mathbb{R}_{\leqslant 0}}
\def\C{\ensuremath{\mathbb{C}}}
\def\U{\ensuremath{\mathcal{U}}}
\def\I{\mathcal{I}}
\def\N{\mathbbN}

\def\P{\ensuremath{\mathbb P}} %{\ensuremath{\mathrm{I\kern-.2em P}}}
%\def\Q{\ensuremath{\mathrm{I\kern-.2em Q}}} % looks strange?
\def\Q{\mathbb{Q}}
\def\E{\ensuremath{\mathbb E}} %{\ensuremath{\mathrm{I\kern-.2em E}}}

\def\hP{\ensuremath{\widehat{\mathrm{I\kern-.2em P}}}}
\def\hE{\ensuremath{\widehat{\mathrm{I\kern-.2em E}}}}

\def\bP{\ensuremath{\overline{\mathrm{I\kern-.2em P}}}}
\def\bE{\ensuremath{\overline{\mathrm{I\kern-.2em E}}}}

\def\bphi{\overline{\phi}}
\def\bpsi{\overline{\psi}}

\def\ott{{0\leq t\leq T}}
\def\idd{{1\le i\le d}}

\def\icc{\mathpzc{i}}
\def\ecc{\mathbf{e}_\mathpzc{i}}

\def\uk{u_{k+1}}
\def\vk{v_{k+1}}

\def\e{\mathrm{e}}
\def\a{\mathrm{a}}
\def\b{\mathrm{b}}
\def\ud{\ensuremath{\mathrm{d}}}
\def\dt{\ud t}
\def\dr{\ud r}
\def\ds{\ud s}
\def\dx{\ud x}
\def\dy{\ud y}
\def\dv{\ud v}
\def\du{\ud u}
\def\dw{\ud w}
\def\dz{\ud z}
\def\dsdx{\ensuremath{(\ud s, \ud x)}}
\def\dtdx{\ensuremath{(\ud t, \ud x)}}

\def\rx{\mathrm{x}}

\def\1{\boldsymbol1}
\def\i{\mathrm{i}}
\def\wf{\widehat f}

\def\tc{\ensuremath{\mathpzc{T}}}
\def\afflm{\ensuremath{(X,\mathcal{X},T_N,u,v)}\xspace}

\def\lsnc{\ensuremath{\mathrm{LSNC-}\chi^2}}
\def\nc{\ensuremath{\mathrm{NC-}\chi^2}}


%% ========================================================================== %%
%% ========================================================================== %%

\newcommand{\cG}{{\mathcal{G}}}
\newcommand{\cH}{{\mathcal{H}}}
\newcommand{\cK}{{\mathcal{K}}}
\newcommand{\cM}{{\mathcal{M}}}
\newcommand{\cT}{{\mathcal{T}}}
\newcommand{\ha}{{\mathbb{H}}}
\newcommand{\indik}{{\mathbf1}}
\newcommand{\ifdefault}[1]{\ensuremath{\mathbf1_{\{\tau \leq #1\}}}}
\newcommand{\ifnodefault}[1]{\ensuremath{\mathbf1_{\{\tau > #1\}}}}

\newcommand\bovermat[2]{%
	\makebox[0pt][l]{$\smash{\overbrace{\phantom{%
					\begin{matrix}#2\end{matrix}}}^{#1}}$}#2}
\newcommand\bundermat[2]{%
	\makebox[0pt][l]{$\smash{\underbrace{\phantom{%
					\begin{matrix}#2\end{matrix}}}_{#1}}$}#2}
\makeatletter



\newcommand{\robert}[1]{\todo[linecolor=blue!40!white,backgroundcolor=blue!20!white,size=\footnotesize]{Robert: #1}}

\def\red{\color{red}}
\def\blue{\color{magenta}}
\def\green{\color{OliveGreen}}

\newcommand{\rr}[1]{{\red #1}}
\newcommand{\rg}[1]{{\green #1}}
%% ========================================================================== %%
%% ========================================================================== %%
\newcommand{\abs}[1]{\left\lvert #1 \right\rvert}


%%%%%%%%%%%%%%%%%%%%%%%%%%%%%%%%chiheb commands

\newcommand{\ie}{\emph{i.e.}}
\newcommand{\eg}{\emph{e.g.}}
\newcommand{\cf}{\emph{cf.}}
\newcommand{\prob}[1]{\mathrm{P}\left(#1\right)}
\newcommand{\expt}[1]{\mathrm{E}\left[#1\right]}
\newcommand{\expth}[1]{\hat{\mathrm{E}}\left[#1\right]}
\newcommand{\var}[1]{\mathrm{Var}\left[#1\right]}


\newcommand{\rset}{\mathbb{R}}
\newcommand{\nset}{\mathbb{N}}
\newcommand{\zset}{\mathbb{Z}}

\newcommand{\PERIOD}{.}
\newcommand{\COMMA}{,}
\newcommand{\BIGSPACE}{\,\,\,\,\,\,\,}



\newcommand{\Ordo}[1]{{\mathcal{O}}\left(#1\right)}
\newcommand{\ordo}[1]{{o}\left(#1\right)}