

One of the numerical challenges encountered in the simulation of the rBergomi dynamics  is the computation of  $\int_{0}^{T} \sqrt{v_t} dW_t^1$ and $V=\int_{0}^{T} v_t dt$, mainly because of the singularity of the Volterra kernel $K^H(s,t)$ at the diagonal $s = t$. In fact,  one needs to jointly simulate two Gaussian processes $(W_t^1, \widetilde{W}^H_t: 0 \le t \le T)$, resulting in $W^1_{t_1},\dots, W^1_{t_N}$ and $\widetilde{W}^H_{t_1},\dots, \widetilde{W}^H_{t_N}$ along a given time grid $t_1 <\dots < t_N$. In the literature, there are essentially two suggested ways to achieve this:
 \begin{enumerate}
% 	\item[i)] \textbf{Simple Euler discretization}: Euler discretization of the integral \eqref{eq:Volterra process}, defining $\widetilde{W}^H$, together with classical simulation of increments of $W^1$. This is inefficient because the integral is singular and adaptivity may not improve the scheme since the singularity moves with time. For this method, we need an $N$-dimensional random Gaussian input vector to produce one (approximate, inaccurate) sample of $W^1_{t_1},\dots, W^1_{t_N}, \widetilde{W}^H_{t_1},\dots, \widetilde{W}_{t_N}$.
 	
 	\item[i)] \textbf{Covariance based approach (exact simulation)} \cite{bayer2016pricing,bayer2018short}: Given that $W^1_{t_1},\dots, W^1_{t_N}, \widetilde{W}^H_{t_1},\dots, \widetilde{W}_{t_N}$ together form a ($2N$)-dimensional Gaussian random vector with computable covariance matrix, $A$, one can use Cholesky decomposition of the covariance matrix to produce exact samples of $W^1_{t_1},\dots, W^1_{t_N},\widetilde{W}^H_{t_1},\dots, \widetilde{W}_{t_N}$ from $2 N$-dimensional Gaussian random vector as  input. This method is exact but slow. The simulation  requires $\Ordo{N^2}$ flops. Note that the offline cost is $\Ordo{N^3}$ flops.
 	
 	\item[ii)]  \textbf{The hybrid scheme of \cite{bennedsen2017hybrid}}: This scheme uses a different approach, which is essentially based on  Euler discretization  but crucially improved by moment
 	matching for the singular term in the left point rule. It is also
 	inexact in the sense that samples produced here do not exactly have the distribution of $W^1_{t_1},\dots, W^1_{t_N}, \widetilde{W}^H_{t_1},\dots, \widetilde{W}_{t_N}$, however they are much more accurate than samples produced from simple Euler discretization, but much faster than method $(i)$. As in method $(i)$, in this case, we need a $2 N$-dimensional Gaussian random input vector to produce one 	sample of $W^1_{t_1},\dots, W^1_{t_N}, \widetilde{W}^H_{t_1},\dots, \widetilde{W}_{t_N}$.
 \end{enumerate}

In the following, we give more details for each scheme. 



\subsection{The Hybrid Scheme}
The hybrid scheme was developed in \cite{bennedsen2017hybrid} for the simulation
of BSS \eqref{eq:BSS} and TBSS \eqref{eq:TBSS} processes with kernel functions, that are similar to a power function near zero, i.e. $K(x)$ is similar to $x^\alpha$ for some $\alpha \in(-\frac{1}{2},\frac{1}{2})$ when $x>0$ is near zero. 

In this section, we start by introducing the general idea of the hybrid scheme. Then, we illustrate how it can be used in simulating  rBergomi  dynamics.
\subsubsection{General Idea of the Hybrid Scheme}
Before describing the general hybrid scheme, there are two assumptions (conditions) that should be fulfilled by the kernel function $K$ in order to apply the scheme. In fact, we assume that the kernel function $K:(0,\infty)\rightarrow [0,\infty)$ satisfy the following
\begin{enumerate}
\item For some $\alpha \in(-\frac{1}{2},\frac{1}{2})\setminus\{0\}$,
$$ K(x)=x^\alpha L_K(x),\quad x \in (0,1],$$
where $L_K:(0,1] \rightarrow [0,\infty)$ is continuously differentiable, slowly varying at $0$ and bounded away from $0$. Moreover there exists a constant $C>0$ such that the derivative $L^\prime_K$ satisfies
$$ \abs{L_K^\prime(x)} \le C \left(1+x^{-1}\right),\quad x \in (0,1].$$
\item The function $K$ is continuously differentiable on $(0,\infty)$, with derivative $K^\prime$ that is ultimately monotonic and also satisfies $\int_{1}^ \infty K^\prime(x)^2 dx<\infty$.
\end{enumerate}
The hybrid scheme starts by discretizing a TBSS process $X_t$ giving by \eqref{eq:TBSS} on the grid $G_{t}^n=\{t,t-\frac{1}{n},t-\frac{2}{n},\dots\}$ for $n \in \nset$, so that we get
\begin{align}\label{eq:hybrid_scheme_1}
X_t \approx \sum_{k=1}^{N_n} \sigma_{t-\frac{k}{n}} \int_{t-\frac{k}{n}}^{t-\frac{k}{n}+\frac{1}{n}} K(t-s)  dW_s,
\end{align}
 where $N_n \in \nset$ is the truncation parameter and denotes the number of discretization steps, and we assume that $\sigma$ doesn’t vary too much (valid in the rBergomi model since $\sigma$ is constant), therefore constant in each discretization cell.
 
The idea of the hybrid scheme  is to split the kernel function into two parts \ref{eq:kernel approximation_hybrid}, depending on a small parameter $\kappa$\footnote{There are different variants of the hybrid scheme depending on the value of $\kappa$ (see \cite{bennedsen2017hybrid} for more details).} 
\begin{align}\label{eq:kernel approximation_hybrid}
 K(t-s)\approx\begin{cases}
              (t-s)^\alpha L_K\left(\frac{k}{n}\right), \quad t-s \in \left[\frac{k-1}{n},\frac{k}{n}\right] \setminus \{0\}&, \quad \text{if} \: k \le \kappa \\
           K\left(\frac{b_k}{n}\right) , \quad t-s \in \left[\frac{k-1}{n},\frac{k}{n}\right] &, \quad \text{if}\:  k >\kappa,
            \end{cases}
\end{align}
and then discretizing the  $X_t$ process into Wiener integrals of power functions and a Riemann sum, appearing from approximating the kernel by power functions near the origin and step functions elsewhere, resulting in \ref{eq:hybrid_scheme_2}
\begin{align}\label{eq:hybrid_scheme_2}
X_t \approx \sum_{k=1}^\kappa    L_K\left(\frac{k}{n}\right) \sigma_{t-\frac{k}{n}} \int_{t-\frac{k}{n}}^{t-\frac{k}{n}+\frac{1}{n}} (t-s)^\alpha   dW_s+ \sum_{k=\kappa+1}^{N_n}  L_K\left(\frac{k}{n}\right)  K\left(\frac{b_k}{n}\right) \sigma_{t-\frac{k}{n}} \int_{t-\frac{k}{n}}^{t-\frac{k}{n}+\frac{1}{n}}   dW_s \PERIOD
\end{align}
\begin{remark}
$\left(b_k\right)_{k=\kappa+1}^\infty$ is a sequence of real numbers that only needs to satisfy  $b_k \in \left[k-1, k\right]\setminus \{0\}$. But there is an optimal sequence $\left(b_k^\ast\right)_{k=\kappa+1}^\infty$, that
minimizes the asymptotic MSE (see \cite{bennedsen2017hybrid} for more details) and it is given by
$$ b_k^\ast=\left(\frac{k^{\alpha+1}-(k-1)^{\alpha+1 }}{\alpha+1}\right)^{\frac{1}{\alpha}},\quad k \ge \kappa+1 \PERIOD$$
\end{remark}

\subsubsection{The hybrid Scheme in the rBergomi Context}
In the context of the rBergomi model, we have $\sigma=1$, $\alpha=H-\frac{1}{2}$ and we can  use the hybrid scheme to simulate $\widetilde{W}^H_t$, which is  a TBSS process, with the kernel function $K^H$ given by \eqref{eq:kernel_rbergomi}. In this context, we can show that by choosing $L_{K^H}(x)=1$, we have  $K^H$ satisfy the conditions required to apply the hybrid scheme  (see \cite{bennedsen2017hybrid}) and therefore we end up with
the following scheme on equidistant grid $\{0,\frac{1}{n},\frac{2}{n},\dots,\frac{nT}{n}\}$
\begin{align}\label{eq:Hybrid_scheme_pre}
\widetilde{W}^H_{\frac{i}{n}} \approx \bar{W}^H_{\frac{i}{n}}&= \sqrt{2H} \left(  \sum_{k=1}^{\min(i,\kappa)} \int_{\frac{i}{n}-\frac{k}{n}}^{\frac{i}{n}-\frac{k}{n}+\frac{1}{n}} \left(\frac{i}{n}-s\right)^\alpha dW_s+\sum_{k=\kappa+1}^{i} \left(\frac{b_k}{n}\right)^{\alpha}  \int_{\frac{i}{n}-\frac{k}{n}}^{\frac{i}{n}-\frac{k}{n}+\frac{1}{n}} dW_s \right) \COMMA
\end{align}
which results for $\kappa=1$  in \eqref{eq:Hybrid_scheme}.
\begin{align}\label{eq:Hybrid_scheme}
\widetilde{W}^H_{\frac{i}{N}} \approx \bar{W}^H_{\frac{i}{N}}&= \sqrt{2H} \left(  W^2_i+\sum_{k=2}^{i} \left(\frac{b_k}{N}\right)^{H-\frac{1}{2}} \left(W_{\frac{i-(k-1)}{N}}^1-W_{\frac{i-k}{N}}^1\right)\right)\COMMA
\end{align}
where $N$ is the number of time steps and 
$$ b_k=\left(\frac{k^{H+\frac{1}{2}}-(k-1)^{H+\frac{1}{2} }}{H+\frac{1}{2}}\right)^{\frac{1}{H-\frac{1}{2}}} \PERIOD$$
The sum in \eqref{eq:Hybrid_scheme} requires the most computational effort in the simulation. Given that \eqref{eq:Hybrid_scheme} can be seen as discrete convolution  (see \cite{bennedsen2017hybrid}), we employ the fast Fourier transform to evaluate it, which results in  $\Ordo{N \log N}$ floating point operations.

We note that the variates $\bar{W}_0^{H},\bar{W}_1^{H},\dots,\bar{W}_{\frac{[Nt]}{N}}^{H}$ are  generated by sampling $[Nt]$ i.i.d draws from a $(\kappa+1)$-dimensional Gaussian distribution and computing a discrete convolution. We denote these pairs  of Gaussian random variables from now on by $(\mathbf{W}^{(1)},\mathbf{W}^{(2)})$.
\subsection{The Exact Scheme Coupled with Hierarchical Representation }\label{sec:The Exact Scheme}
The exact scheme is based on simulating exactly the joint covariance of $(\widetilde{W}^H,W^1)$, which is given by
\begin{align}\label{eq:joint covariance cholesky}
 \begin{cases}
               \text{Var}(\widetilde{W}_t)=t^{2H}&,\quad t \ge 0\\
                \text{Cov}(\widetilde{W}_t,\widetilde{W}_s)=t^{2H} G\left(\frac{s}{t}\right)&,\quad s>t \ge 0\\
              \text{Cov}(\widetilde{W}_t,W^1_s)=\rho D_H \left(t^{H+\frac{1}{2}}-\left( t-\min(t,s)^{H+\frac{1}{2}} \right) \right) &,\quad s,t \ge 0\\
              \text{Cov}(W^1_t,W^1_s)=\min(t,s)&,\quad s,t \ge 0\COMMA
            \end{cases}
\end{align}
with 
$$ D_H=\frac{\sqrt{2H}}{H+\frac{1}{2}}$$
and G is given by \eqref{eq:correlation_tilde_W_fun} for $x \ge 1$ and $\gamma=\frac{1}{2}-H$.

Different ways can be used to compute  \eqref{eq:correlation_tilde_W_fun}. For instance, in the literature we find that  in \cite{bayer2018short}
\begin{equation}
G_2(x)=2H \left(  \frac{1}{1-\gamma} x^{-\gamma}+\frac{\gamma}{1-\gamma}x^{-(1+\gamma)} \frac{1}{2-\gamma} {}_2F_1\left(1,1+\gamma,3-\gamma,x^{-1} \right)  \right),
\end{equation}
where 
\begin{align*}
{}_2F_1(a,b,c;z)=\frac{\Gamma(c)}{\Gamma(b) \Gamma(c-b)} \int_{0}^1 t^{b-1} (1-t)^{c-b-1} (1-tz)^{-a} dt.
\end{align*}
\begin{remark}
Because the covariance matrix does not have any $0$ values, if $\rho\neq0$, the computational effort for the exact simulation is quite high and the algorithm is very slow.
\end{remark}
Let us denote by the matrix $A$, the computable covariance matrix of  $ \widetilde{W}^H_{t_1},\dots, \widetilde{W}_{t_N},W^1_{t_1},\dots, W^1_{t_N}$ using \eqref{eq:joint covariance cholesky}.  We can use Cholesky decomposition of $A$ to produce exact samples of $W^1_{t_1},\dots, W^1_{t_N}, \widetilde{W}^H_{t_1},\dots, \widetilde{W}_{t_N}$.

In fact let us denote by $L$ the triangular matrix resulting from Cholesky decomposition such that 
\[
L=
\left(
\begin{array}{c|c}
L_1& 0 \\
L_2 & L_3
\end{array}
\right),
\]
where $L_1, L_2,L_3$ are $N \times N$ matrices, such that $L_1$ and $L_3$ are triangular.

Then, given  a $2 N \times 1$-dimensional Gaussian random input vector, $\mathbf{X}=(X_1, \dots,X_N, X_{N+1}, \dots, X_{2N})'$, we have

\begin{align}
\mathbf{W}^{(1)}=L_1 \mathbf{X}_{1:N}, \quad \widetilde{\mathbf{W}}= 
\left(
\begin{array}{c|c}
L_2 & L_3 
\end{array}
\right) \mathbf{X}.
\end{align}
On the other hand, let us assume that we can construct $\mathbf{W}^{(1)}$ hierarchically  through  Brownian bridge construction defined by the linear mapping given by the matrix $G$, then given a $ N$-dimensional Gaussian random input vector, $\mathbf{Z}^\prime$, we can write
\begin{align*}
\mathbf{W}^{(1)}=G  \mathbf{Z}^\prime \COMMA
\end{align*}
and consequently
\begin{align*}
 \mathbf{X}_{1:N}= L_1^{-1} G  \mathbf{Z}^\prime \PERIOD
\end{align*}
Therefore, given a $2 N$-dimensional Gaussian random input vector, $\mathbf{Z}=(\mathbf{Z}^\prime,\mathbf{Z}^{\prime \prime})$, we define our hierarchical representation by
\begin{align}\label{eq: Construction}
\mathbf{X}=\left(
\begin{array}{c|c}
L_1^{-1} G & 0\\
0 & I_{N} 
\end{array}
\right) \mathbf{Z}.
\end{align}

As described in Section 4 in \cite{bayer2016pricing}, the exact scheme is given by the following. For $N$ number of time steps and $M$ the number of simulations,

\begin{enumerate}
\item Construct the joint covariance matrix for the Volterra process $\tilde{W}$ and the Brownian motion $W^1$ and compute its Cholesky decomposition.
\item For each time, generate iid normal random vectors and multiply them
by the lower-triangular matrix obtained by the Cholesky decomposition
to get a $M \times 2 N$ matrix of paths of $\tilde{W}$ and $W^1$ with the correct joint marginals.
\item  With these paths held in memory, we may evaluate the expectation
under of any payoff of interest.
\end{enumerate}
%We need to make sure that $\mathbf{X}$  has Gaussian distribution as an outcome of the construction \eqref{eq: Construction}. Consequently, we need to compute carefully $L_1^{-1}$. Actually, I observed that $L_1=I_{N \times N}.$ Therefore,  $\mathbf{X}$  has Gaussian distribution as an outcome of the construction \eqref{eq: Construction}.

