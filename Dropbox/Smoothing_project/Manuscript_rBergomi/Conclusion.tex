In this work,  we propose  novel, fast option pricers,  for options whose underlyings  follow the rBergomi model as in \cite{bayer2016pricing}.  The new methods  are based on hierarchical deterministic quadrature methods:  i) ASGQ specifically using the same construction in \cite{haji2016multi}, and ii) QMC method. Both techniques are coupled with Brownian bridge construction and Richardson extrapolation.

In this work,  we propose  novel, fast option pricers,  for options whose underlyings  follow the rBergomi model as in \cite{bayer2016pricing}.  The new methods  are based on hierarchical deterministic quadrature methods:  i) ASGQ as in \cite{haji2016multi}, and ii) QMC method. Both techniques are coupled with Brownian bridge construction and Richardson extrapolation.
Given that the only prevalent option, in this context, is to use different variants of the MC method, which is computationally expensive, our first contribution  is that we uncover the available regularity in the rBergomi model and  design novel  approaches based on an ASGQ and QMC. These approaches  open a new research direction in this field to investigate the performance of other methods besides MC, for pricing and calibrating under the rBergomi model. Our second contribution is that we reduce the computational cost  through bias reduction by using Richardson extrapolation. Finally, assuming one targets price estimates with a sufficiently small relative error tolerance, our proposed method demonstrates substantial computational gains  over the standard MC method, when pricing under the rBergomi model, even for very small values of the Hurst parameter. We show  these gains through our numerical experiments for  different parameter constellations.  We clarify that we do not claim that these gains will hold in the asymptotic regime, i.e.  for higher accuracy requirements. Furthermore, the use of Richardson extrapolation is justified in the pre-asymptotic regime, in which our observed experimental results suggest a convergence of order one for the weak error. We emphasize that, to the best of our knowledge, no proper weak error analysis has been done in the rough volatility context. 

In this work, we limit ourselves to compare our novel proposed method against the standard MC. A more systematic comparison against the variant of MC proposed in \cite{mccrickerd2018turbocharging}  can be carried out but this remains for a future study. Finally, accelerating  our novel  methods can be achieved  by coupling ASGQ and QMC with a more optimal hierarchical path generation method than Brownian bridge construction, such as PCA or LT transformations.