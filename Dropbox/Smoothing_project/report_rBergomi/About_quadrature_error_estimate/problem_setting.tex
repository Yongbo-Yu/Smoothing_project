

\subsection{The rBergomi model}\label{sec:The rBergomi model}

We consider the rBergomi model for the price process $S_t$ as defined in  \cite{bayer2016pricing}, normalized to $r=0$\footnote{$r$ is the interest rate.}, which is defined by

\begin{align}\label{eq:rBergomi_model1}
	dS_t &= \sqrt{v_t} S_t dZ_t, \nonumber \\
	v_t &= \xi_0(t) \exp\left( \eta \widetilde{W}_t^H - \frac{1}{2} \eta^2 t^{2H} \right),
\end{align}
where the Hurst parameter $0 < H < 1$  and  $\eta>0$. We refer to $v_t$ as the variance process, and $\xi_0(t) = \expt{v_t}$ is  the forward variance curve.  Here, $\widetilde{W}^H $ is a certain Riemann-Liouville fBm
process\footnote{The so-called Riemann-Liouville processes are deduced from the standard Brownian motion by applying Riemann-Liouville fractional operators, whereas the standard fBm requires a weighted fractional operator \cite{marinucci1999alternative,picard2011representation}.},  defined by
\begin{align}\label{eq:Volterra process}
	\widetilde{W}_t^H = \int_0^t K^H(t,s) dW_s^1, \quad t \ge 0 \COMMA
\end{align}
where the kernel $K^H : \rset_+  \rightarrow \rset_+$ is
\begin{equation}\label{eq:kernel_rbergomi}
 \quad K^H(t-s) = \sqrt{2H} (t-s)^{H - 1/2},\quad \forall \: 0 \le s \le t.
\end{equation}
By construction, $\widetilde{W}^H $ is a centered, locally $(H-\epsilon)$- H\"older continuous, Gaussian process with $\text{Var}\left[\widetilde{W}^H_t \right] = t^{2H}$, and a dependence structure defined by 
 \begin{equation*}
 \expt{\widetilde{W}^H_u  \widetilde{W}^H_v}=u^{2H} G\left(\frac{v}{u} \right),\quad v >u \COMMA
 \end{equation*}
 where for $x \ge 1$ and $\gamma=\frac{1}{2}-H$
 \begin{equation}\label{eq:correlation_tilde_W_fun}
G(x)=2H \int_{0}^1 \frac{ds}{(1-s)^{\gamma} (x-s)^{\gamma}}.
 \end{equation}
%\red{We note that $\widetilde{W}$ is also a Brownian semi-stationary (BSS) process (see Definition \ref{def:semi-stationary process}), which was introduced by Barndorff-Nielsen and Schmiegel \cite{barndorff2007ambit,barndorff2009brownian}.
%
%\begin{definition}[Semi-stationary process]\label{def:semi-stationary process}
%$X_t$ is called a \textit{Brownian semi-stationary} process if
%\begin{equation}\label{eq:BSS}
%X_t=\int_{-\infty}^t K(t-s) \sigma_s dW_s\COMMA
%\end{equation}
%for some deterministic kernel function $K$ and an adapted intermittency process $\sigma$. If the integral starts at $0$ instead of $−\infty$
%\begin{equation}\label{eq:TBSS}
%X_t=\int_{0}^t K(t-s) \sigma_s dW_s\COMMA
%\end{equation}
%we call the process \textit{truncated Brownian semi-stationary} process (TBSS).
%\end{definition} 
%} 
In \eqref{eq:rBergomi_model1} and \eqref{eq:Volterra process}, $W^1, Z$ denote two \emph{correlated} standard Brownian motions with correlation $\rho \in ]-1,0]$, so that we can represent $Z$ in terms of $W^1$ as
\begin{align*}
	Z=\rho	W^1+ \bar{\rho}W^\perp = \rho W^1+\sqrt{1-\rho^2} W^\perp,
\end{align*}
where $(W^1,W^\perp)$ are two independent standard Brownian motions.
Therefore, the solution to \eqref{eq:rBergomi_model1}, with $S(0)=S_0$, can be written as 

\begin{align}\label{eq:rBergomi_model}
	S_t&= S_0  \operatorname{exp}\left( \int_{0}^{t} \sqrt{v(s)} dZ(s)- \frac{1}{2} \int_{0}^{t} v(s) ds   \right),\quad S_0>0 \nonumber\\
	v_u&=\xi_0(u) \operatorname{exp}\left( \eta \widetilde{W}_u^H- \frac{\eta^2}{2} u^{2H} \right), \quad \xi_0>0 \PERIOD
\end{align}
The filtration $(\mathcal{F}_t)_{t\ge 0}$ can here be taken as the one generated by the two-dimensional Brownian motion $(W^1,W^\perp)$ under the risk neutral measure $\mathbb{Q}$, resulting in  a filtered probability space $(\Omega,\mathcal{F}, \mathcal{F}_t,\mathbb{Q})$. The stock price process $S$ is clearly then a local
$(\mathcal{F}_t)_{t\ge 0}$-martingale and a supermartingale.  We shall henceforth use the notation $\expt{.} = E^{\mathbb{Q}}\left[. \mid \mathcal{F}_0\right]$ unless we state otherwise.
\begin{remark}
The rBergomi model is non-Markovian in the instantaneous variance $v_t$, that is $E^{\mathbb{Q}}\left[v_u\mid \mathcal{F}_t\right] \neq= E^{\mathbb{Q}}\left[v_u\mid v_t\right]$. However, it is Markovian in the state vector by definition, that is $E^{\mathbb{Q}}\left[v_u\mid\mathcal{F}_t\right]=\xi_t(u)$.
\end{remark}
\subsection{Option pricing under the rBergomi model}\label{sec:Option pricing under rBergomi model}

We are interested in pricing European call options under the rBergomi model. Assuming $S_0 = 1$, and using the conditioning argument on the $\sigma$-algebra generated by $W^1$ (an argument first used by \cite{romano1997contingent} in the context of Markovian stochastic volatility  models), we can  show that the call price is given by

\begin{align}\label{BS_formula_rbergomi}
	C_{\text{RB}}\left( T, K \right) &= \text{E}\left[ \left(S_T - K \right)^+ \right]  \nonumber\\
	&=\expt{\expt{(S_T-K)^+ \mid \sigma(W^1(t) ,t \le T)}}\nonumber \\
	&=\text{E}\left[C_{\text{BS}}\left( S_0 = \operatorname{exp}\left(\rho \int_0^T \sqrt{v_t} dW_t^1 - \frac{1}{2}
	\rho^2 \int_0^T v_t dt\right),\ k = K, \ \sigma^2 = (1-\rho^2)
	\int_0^T v_t dt \right) \right],
\end{align}
where $C_{\text{BS}}(S_0,k,\sigma^2)$ denotes the Black-Scholes call price, for initial spot price $S_0$, strike price $k$ and volatility $\sigma^2$.

%To show \eqref{BS_formula_rbergomi}, we use the orthogonal decomposition of $S_t$ into $S_{t}^1$ and $S_{t}^2$, where
%\begin{align*}
%	S_t^1=\mathcal{E}\{ \rho \int_{0}^{t}  \sqrt{v_s} dW_s^1\}, \: S_t^2= \mathcal{E}\{ \sqrt{1-\rho^2} \int_{0}^{t}  \sqrt{v_s} dW_s^\perp  \}	,
%\end{align*}
%and $\mathcal{E}(.)$ denotes the stochastic exponential; then, if we define $\mathcal{F}_t^1= \sigma\{ W_s^1: s\le t\}$, we obtain by conditional log-normality
%\begin{align*}
%	\log S_t \mid \mathcal{F}_t^1 \sim \mathcal{N}\left( \log S_t^1-\frac{1}{2} (1-\rho^2) \int_{0}^{t} v_s ds , (1-\rho^2) \int_{0}^{t} v_s ds \right),
%\end{align*} 
%and by using the same procedure as in \cite{romano1997contingent} we obtain \eqref{BS_formula_rbergomi}.

We point out that the analytical smoothing, based on conditioning, performed in \eqref{BS_formula_rbergomi} enables us to uncover the available regularity, and hence  get a smooth, analytic integrand inside the expectation. Therefore, applying a deterministic quadrature technique such as ASGQ or QMC becomes an adequate option for computing the call price as we will investigate later. A similar conditioning was used in \cite{mccrickerd2018turbocharging} but for variance reduction purposes only.

