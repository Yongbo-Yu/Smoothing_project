We recall that our goal is to compute the expectation in \eqref{BS_formula_rbergomi}.  We need   $2N$-dimensional Gaussian inputs for the used  hybrid  scheme ($N$ is the number of time steps in  the time grid), namely
\begin{itemize}
	\item $\mathbf{W}^{(1)}=\{W^{(1)}_i\}_{i=1}^N$: The $N$ Gaussian random variables that are defined in Section  \ref{sec:The rBergomi model}.
	\item $\mathbf{W}^{(2)}=\{W^{(2)}_j\}_{j=1}^N$: An artificially introduced $N$ Gaussian random variables that are used for left-rule points in the hybrid scheme.
\end{itemize}
We can rewrite \eqref{BS_formula_rbergomi} as 
\begin{align}\label{BS_formula_rbergomi_2}
C_{\text{RB}}\left( T, K \right)&=\text{E}\left[C_{\text{BS}}\left( S_0 = \operatorname{exp}\left(\rho \int_0^T \sqrt{v_t} dW_t^1 - \frac{1}{2}
\rho^2 \int_0^T v_t dt\right),\ k = K, \ \sigma^2 = (1-\rho^2)
\int_0^T v_t dt \right) \right] \nonumber \\
&\approx \int_{\rset^{2N}} C_{BS} \left(G(\mathbf{w}^{(1)},\mathbf{w}^{(2)})\right) \rho_{N}(\mathbf{w}^{(1)})  \rho_{N}(\mathbf{w}^{(2)}) d\mathbf{w}^{(1)} d\mathbf{w}^{(2)} \nonumber \\
&:=C_{RB}^{N},
\end{align}
where $G$  maps  $2N$ independent standard Gaussian random inputs to the parameters fed to Black-Scholes formula, and  $\rho_N$ is the multivariate Gaussian density, given by 
\begin{equation*}\label{eq: multivariate gaussian distribution}
\rho_N(\mathbf{z})=\frac{1}{(2 \pi)^{N/2}} e^{-\frac{1}{2} \mathbf{z}^T \mathbf{z}} \PERIOD
\end{equation*} 
Therefore, the initial integration problem that we are solving lives in $2N$-dimensional space, which becomes very large as the number of time steps $N$, used in the hybrid scheme, increases.

If we denote by $\mathcal{E}_{\text{tot}}$ the total error of approximating the  expectation in \eqref{BS_formula_rbergomi} using the ASGQ estimator, $Q_N$, then we have a natural error decomposition
\begin{align}\label{eq:total_error_ASGQ}
\mathcal{E}_{\text{tot}} & \le \abs{C_{\text{RB}}-C_{\text{RB}}^N}+\abs{C_{\text{RB}}^N-Q_{N}} \le \mathcal{E}_B(N)+ \mathcal{E}_Q(\text{TOL}_{\text{ASGQ}},N),
\end{align}
where  $\mathcal{E}_Q$ is the quadrature error, $\mathcal{E}_B$  is the bias, $\text{TOL}_{\text{ASGQ}}$ is a user selected tolerance for ASGQ method, and $C_{\text{RB}}^N$ is the biased price computed with $N$ time steps as given by \eqref{BS_formula_rbergomi_2}.

On the other hand, the total error of approximating the  expectation in \eqref{BS_formula_rbergomi} using the randomized QMC or MC estimator, $Q^{\text{MC(QMC)}}_N$ can be bounded by

\begin{align}\label{eq:total_error_MC}
	\mathcal{E}_{\text{tot}} & \le \abs{C_{\text{RB}}-C_{\text{RB}}^N}+\abs{C_{\text{RB}}^N-Q^{\text{MC (QMC)}}_N} \le \mathcal{E}_B(N)+ \mathcal{E}_{S}(M,N),
\end{align}
where  $\mathcal{E}_S$ is the statistical error\footnote{The statistical error estimate of MC or randomized QMC is  $C_{\alpha} \frac{\sigma_M}{\sqrt{M}}$, where $M$ is the number of samples and $C_{\alpha}=1.96$ for $95\%$ confidence interval.}, $M$ is the number of samples used for MC or randomized QMC method.











