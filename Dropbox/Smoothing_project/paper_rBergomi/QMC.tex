In this work, we use QMC to compare with MC and ASGQ. Specifically, we use the lattice rules family of QMC \cite{sloan1985lattice,cools2008belgian,nuyens2014construction}.  The main input for the lattice rule is one integer vector with $d$ component ($d$ dimension of the integration problem).

In fact, given an integer vector $z = (z_1,\dots, z_d)$ known as \textit{the generating vector}, a (rank-$1$) lattice rule with $n$ points takes the form

\begin{equation}
Q_n(f):=\frac{1}{n}\sum_{k=0}^{n-1} f \left( \frac{kz \: \text{mod}\: n}{n}\right).
\end{equation}
The quality of the lattice rule depends on the choice of the generating vector. Due to the modulo operation, it suffices to consider the values from $1$ up to $n-1$, leaving out $0$ which is clearly a bad choice. Furthermore, we restrict the values to those relatively prime to $n$, to ensure that every one-dimensional projection of the $n$ points yields $n$ distinct values. Thus, we write $\mathbf{z} \in \mathbb{U}_n^d$, with $\mathbb{U}_n:=\{z \in zset: 1 \le z \le n-1\: \text{and gcd}(z,n)=1\}$.

For theoretical analysis one often assumes that $n$ is prime to simplify some
number theory arguments. For practical application,  $n$ is often taken  to be a power of $2$. The total number of possible choices for the generating vector is then $(n-1)^d$ and $(n/2)^d$ , respectively. Even if we have a criterion to assess the quality of the generating vectors, there are simply too many choices to carry out an exhaustive search when $n$ and $d$ are large. 


To get an unbiased approximation of the integral, we use  a randomly shifted lattice rule, which also allows us to obtain a practical error estimate in the same way as the MC method. It works as follows. We generate $q$ independent random shifts $\Delta^{(i)}$ for $i=0,\dots,q-1$ from the uniform distribution on $[0,1]^d$ . For the same fixed lattice generating vector $z$, we compute the $q$ different shifted lattice rule approximations and denote
them by $Q^{(i )}_n(f)$ for $i=0,\dots,q-1$.We then take the average
\begin{align}
\overline{Q}_{n,q}(f)=\frac{1}{q} \sum_{i=0}^{q-1}Q^{(i )}_n(f)=\frac{1}{q}\sum_{i=0}^{q-1}\left(\frac{1}{n}\sum_{k=0}^{n-1} f \left( \frac{kz+\Delta^{(i)}  \: \text{mod}\: n}{n}\right)  \right)
\end{align}
as our final approximation to the integral.

We note that since we are dealing with Gaussian randomness and with integrals in infinite support, we use the inverse of the standard normal cumulative distribution function as a pre-transformation to map the problem to $[0,1]$ and then use QMC.

In our numerical test, we use a pre-made point generators using latticeseq\_b2.py in python from   \url{https://people.cs.kuleuven.be/~dirk.nuyens/qmc-generators/}.


