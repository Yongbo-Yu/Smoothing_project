Modeling the volatility to be stochastic, rather than deterministic as in the Black-Scholes model, enables quantitative analysts to  explain certain phenomena observed in option price data, in particular the implied volatility smile. However, this family of models has a  main drawback in failing  to capture the true steepness of the implied volatility smile close to maturity. Jumps can be added to stock price models to overcome this undesired feature, for instance by modeling the stock price process as an exponential L\'evy process. However, the addition of jumps to stock price processes remains controversial \cite{christensen2014fact,bajgrowicz2015jumps}. 

Motivated by the statistical analysis of realized volatility by Gatheral, Jaisson and Rosenbaum \cite{gatheral2018volatility} and the theoretical results on implied volatility    \cite{alos2007short,fukasawa2011asymptotic}, rough stochastic volatility has emerged as a new paradigm in quantitative finance, overcoming the observed limitations of  diffusive stochastic volatility models. In these models, the trajectories of the volatility  have lower H\"older regularity than the trajectories of standard Brownian motion \cite{bayer2016pricing,gatheral2018volatility}. In fact, they are based on fractional Brownian motion (fBm), which  is a centered Gaussian process, whose covariance structure depends on  the so-called Hurst parameter, $H$ (we refer to  \cite{mandelbrot1968fractional,coutin07introduction,biagini2008stochastic} for more details regarding the fBm processes). In the rough volatility case, where $0<H<1/2$, the fBm has negatively correlated increments and rough sample paths.   Gatheral, Jaisson, and Rosenbaum \cite{gatheral2018volatility}  empirically demonstrate the advantages of such models. For instance, they show that the log-volatility in practice has a similar behavior to  fBm with the Hurst exponent $H \approx 0.1$ at any reasonable time scale (see also  \cite{gatheral2014volatility_2}).  These results were confirmed  by Bennedsen, Lunde and Pakkanen \cite{bennedsen2016decoupling}, who studied over a thousand individual US equities and showed that $H$ lies in $(0,1/2)$ for each equity. Other  works \cite{bennedsen2016decoupling,bayer2016pricing,gatheral2018volatility} showed further benefits of  such rough volatility models over  standard stochastic volatility models,   in terms of explaining crucial phenomena  observed in  financial markets.
 
One of the first rough volatility models is the rough Bergomi (rBergomi) model, developed by Bayer, Friz and Gatheral \cite{bayer2016pricing}. This model showed   consistent behavior with the stylized fact of implied volatility surfaces being essentially time-invariant. It was also observed that this model is able to capture the term structure of skew observed in equity markets. The construction of the rBergomi model was performed by  moving from a physical to a pricing measure and by simulating prices under that model to fit  the implied volatility surface well in the case of the S\&P $500$ index with few parameters. The model may be seen as a non-Markovian extension of the Bergomi variance curve model \cite{bergomi2005smile}.
 
Despite the promising features of the rBergomi model, pricing  and hedging under such a model still constitutes a challenging and time-consuming task due  to the non-Markovian nature of the fractional driver.  In fact, the standard numerical pricing methods, such as: PDE discretization schemes, asymptotic expansions and transform
methods, although efficient in the case of diffusion, are not easily  carried over to the rough setting. Furthermore,  due to the lack of Markovianity and affine structure, conventional analytical pricing methods  do not apply. To the best of our knowledge, the only prevalent method for pricing  options under such models is Monte Carlo (MC) simulation. In particular,  recent advances in simulation methods for the rBergomi model and different variants of pricing methods based on  MC under such a model   have been proposed in \cite{bayer2016pricing,bayer2017regularity,bennedsen2017hybrid,mccrickerd2018turbocharging,jacquier2018vix}.  For instance, in \cite{mccrickerd2018turbocharging}, the authors employ a novel composition of variance reduction methods. When pricing under the rBergomi model, they achieved substantial computational gains  over the standard MC method.  Greater   analytical understanding of option pricing and implied volatility under this model has been achieved  in \cite{jacquier2017pathwise,forde2017asymptotics,bayer2018short}.  It is crucial to note that hierarchical variance reduction methods, such
as Multi-level Monte Carlo (MLMC), are inefficient in this context, because of the poor behavior of the strong error, that is of the order of $H$ \cite{neuenkirch2016order}.

Despite the recent advances in the MC method, pricing under the rBergomi model is still a time-consuming task. To overcome this issue,  we design  a novel fast option pricer in this work,  based on a  \red{i) adaptive sparse grids quadrature (ASGQ), specifically using the same construction in \cite{haji2016multi}, and ii) Quasi Monte Carlo (QMC). Both techniques are coupled with Brownian bridge construction and Richardson extrapolation, for options whose underlyings  follow the rBergomi model.  To use these two deterministic quadrature techniques (ASGQ and QMC) for our purposes}, we  solve two main issues that constitute the two stages of our newly designed method. In the first stage, we smoothen the integrand by using the conditional expectation as was proposed in \cite{romano1997contingent}, in the context of Markovian stochastic volatility  models, and in \cite{bayersmoothing}, in the context of basket options.   In a second stage, we apply \red{the deterministic quadrature method}, to solve the integration problem. In this stage, we apply two transformations before using \red{the ASGQ or QMC method}, to overcome the issue of facing a high-dimensional integrand due to the discretization scheme used for simulating the rBergomi dynamics. Given that \red{ASGQ and QMC} benefit from anisotropy, the first transformation consists of applying a hierarchical  path generation method, based on Brownian
bridge (Bb) construction, with the aim of reducing the effective dimension. The second transformation consists of applying Richardson extrapolation to reduce the bias, which in turn reduces the needed number of time steps in the coarsest level to achieve a certain error tolerance and consequently  the maximum number of dimensions needed for the integration problem. We emphasize that we are interested in  the pre-asymptotic regime (corresponding to a small number of time steps), and the use of Richardson extrapolation is justified by our observed experimental results in that regime,  which suggest, in particular, that we have convergence of order one for the weak error and  that the pre-asymptotic regime is enough to get sufficiently accurate estimates for the option prices. Furthermore, we emphasize that no proper weak error analysis has been done in the rough volatility context.

Our first contribution is that we design a novel alternative approach based on \red{ a deterministic quadrature}, in contrast to the aforementioned studies such as \cite{mccrickerd2018turbocharging}. Given that the only prevalent option in this context is to use different variants of the MC method, our work opens a new  research direction in this field, i.e.  to investigate the performance of methods other than MC for pricing and calibrating under the rBergomi model. Our second contribution is that we reduce the computational cost  through bias reduction by using Richardson extrapolation. Finally, assuming one targets price estimates with a sufficiently small  error tolerance, our proposed method demonstrates substantial computational gains  over the standard MC method, even for very small values of  $H$. We show  these gains through our numerical experiments for  different parameter constellations. However, we do not claim that these gains will hold in the asymptotic regime, which requires higher accuracy. Furthermore,  in this work, we limit ourselves to comparing our novel proposed method against the standard MC. A more systematic comparison with the variant of MC proposed in \cite{mccrickerd2018turbocharging}  can be carried out in future but has not been included in this work. 

The outline of this paper is as follows: We start in Section \ref{sec:Problem setting} by  introducing  the pricing framework that we are considering in this study. We provide some details about the rBergomi model, option pricing under this model and the simulation schemes used to simulate asset prices following the rBergomi dynamics. In Section \ref{sec:Weak error analysis}, we provide a short weak error analysis in the context of the rBergomi  and explain how we choose the optimal simulation scheme for an optimal performance of our approach. Then, in Section \ref{sec:Details our approach and error bounds}, we explain the different building blocks that constitute our proposed methods, which are basically \red{ASGQ, QMC}, Brownian bridge construction, and Richardson extrapolation. Finally, in Section \ref{sec:Numerical tests}, we show the results obtained through the different numerical experiments conducted across different parameter constellations for the rBergomi model. The reported results show the high potential of \red{our proposed methods} in this context.

