Modeling the volatility to be stochastic, rather than constant as in Black-Scholes model, enabled the quantitative finance experts to  explain certain phenomena observed in option price data, in particular the implied volatility smile. However, this family of models has a  main drawback consisting in failing  to capture the true steepness of the implied volatility smile close to maturity. To overcome this undesired feature, experts suggested to add jumps to stock price models, for instance modelling the stock price process as an exponential L\'evy process. Unfortunately, the issue of the presence of jumps in stock price processes remains controversial \cite{christensen2014fact,bajgrowicz2015jumps}, and is one of the major critics for such models. 



Motivated by the statistical analysis of realised volatility by Gatheral, Jaisson and Rosenbaum \cite{gatheral2014volatility} and the theoretical results on implied volatility by Fukasawa \cite{fukasawa2011asymptotic}, rough stochastic volatility has emerged as a new paradigm in quantitative finance, overcoming the limitations observed by  diffusive stochastic volatility models. In these models, the trajectories of volatility  has H\"older regularity lower  than those of the standard Brownian motion \cite{gatheral2014volatility,bayer2016pricing}. In fact, they are based on fractional Brownian motion (fBM), which  is a centred Gaussian process, whose covariance structure depends on the Hurst parameter $0<H<1/2$ (we refer to  \cite{mandelbrot1968fractional,coutin07introduction,biagini2008stochastic} for more details regarding the fBm processes). In this case, the fBM has negatively correlated increments and "rough" sample paths.   Gatheral, Jaisson, and Rosenbaum \cite{gatheral2014volatility} prove  empirically  the advantages of such models. For instance, they show that the log-volatility in practice has a similar behavior as  fBM with the Hurst exponent $H \approx 0.1$ at any reasonable time scale (see also  \cite{gatheral2014volatility_2}).  These results were confirmed  by Bennedsen, Lunde and Pakkanen \cite{bennedsen2016decoupling}, who study over a thousand individual US equities and showed that the Hurst parameter $H$ lies in $(0,1/2)$ for each equity. Other  works showed further benefits of  such rough volatility models over  standard stochastic volatility ones,   in terms of explaining crucial phenomena  observed in  financial markets. For instance, from a statistical point of view, we mention \cite{gatheral2014volatility,bennedsen2016decoupling}, and from an option-pricing point of view, we mention \cite{bayer2016pricing}.
   

 
 One of the recent rough volatility models is the rough Bergomi (rBergomi) model, developed by Bayer, Friz and Gatheral \cite{bayer2016pricing}. This model showed   consistent bahavior with the stylised fact of implied volatility surfaces being essentially time-invariant. It was also proven that this model 
 enables to capture the term structure of skew observed in equity markets. The construction of the rBergomi model was performed by  moving from  physical to pricing measure and simulating prices under that model to fit well the implied volatility surface in the case of the S\&P $500$ index with few parameters. The model 's name is inspired by the Bergomi variance curve model \cite{bergomi2005smile}, and may be seen as a non-Markovian
 extension of the latter.
 
Despite the promising features of the rBergomi model, pricing  and hedging under such model, still constitutes a challenging task, due  to the non-Markovian nature of the fractional driver. In fact, the standard numerical pricing methods, being efficient in  the case of diffusions, such as: Monte Carlo (MC) estimators, PDE discretization schemes, asymptotic expansions and transform
methods, are not easily  carried over to the rough setting. Furthermore,  due to the lack of Markovianity or affine structure, conventional analytical pricing methods  do not apply. To the best of our knowledge, the only prevalent method for pricing  options under such model is MC simulation. In particular,  recent advances in simulation methods for the rough Bergomi model and different variants of pricing methods based on  MC under such model   have been proposed in \cite{bayer2016pricing,bayer2017regularity,mccrickerd2017turbocharging,bennedsen2017hybrid,jacquier2018vix}.  For instance, in \cite{mccrickerd2017turbocharging}, the authors employ a novel composition of variance reduction methods. When pricing under rBergomi model, they got  substatantial computational gains  over the standard MC method. On the other hand,  more attempts to have  analytical understanding of option pricing and implied volatility under this model have been achieved  in \cite{jacquier2017pathwise,bayer2017short,forde2017asymptotics}. We should point out that pricing and model calibration under rough volatility models still remains a time consuming task.


In this paper,  we desing a novel hierarchical fast option pricer,  based on a  hierarchical adaptive sparse grids quadrature, specifically  multi-index stochastic collocation (MISC) as in  \cite{haji2016multi}, coupled with Brownian bridge construction and Richardson extrapolation, for options whose underlyings  follow rBergomi model as in \cite{bayer2016pricing}.  In order to use a  variant of adaptive sparse grids quadrature for our purposes, we had to solve two main issues, that constitutes the two stages of our new designed method. In the first stage, we smoothen the integrand by using the conditional expectation trick as was proposed by \cite{romano1997contingent}, in the context of Markovian stochastic volatility  models.   In a second stage, we applied a variant of hierarchical adaptive sparse grids quadrature, specifically MISC as in \cite{haji2016multi}, to solve the integration problem. In this stage, we had to apply two pre-transformations before using the MISC solver, in order to overcome the issue of facing a high dimensional integrand, due to the discretization scheme used for simulating the rBergomi dynamics. Given that MISC profits from anisotropy, the first pre-transformation consists of applying a hierarchical  path generation method, based on Brownian
bridge (BB) construction, with the aim of reducing the effective dimension. The second pre-transformation consists of applying Richardson extrapolation to reduce the bias, resulting in reducing the needed number of time steps in the coarsest level to achieve a certain error tolerance, and therefore,  the maximum number of dimensions needed for the integration problem.

Compared to the works that we mentioned above, mainly \cite{mccrickerd2017turbocharging}, our first contribution is that we design an alternative approach based on a variant of adaptive sparse grid quadrature. Given that the only prevalent option, in this context, is to use different variants of MC method, our work opens a new persepective for experts in this field to investigate the performance of other methods besides MC, to solve pricing and calibration problems related to rBergomi model. Our second contribution is that we reduce the computational cost not only through variance reduction, as in  \cite{mccrickerd2017turbocharging}, by using the conditional expectation, but also through bias reduction through using Richardson extrapolation. Finally, assuming one targets prices estimates with a suffciently small degree of error tolerance, our thrid contribution is manifested by the observed substantial computational gains  over standard MC method, when pricing under rBergomi model. We show  these gains through our numerical experiments for  different parameters constellations. In this paper, we limit ourselves to compare our novel proposed method against standard MC. A more systematic comparison against the variant of MC proposed in \cite{mccrickerd2017turbocharging}  can be carried over but this is left for a future work. Another eventual direction of research may also investigate the performance of quasi Monte Carlo (QMC) for such problems.


The outline of this paper is as follows: We start in Section \ref{sec:Problem setting} by  introducing  the pricing framework that we are considering. We provide some details about the rBergomi model, option pricing under this model and simulation scheme used to simulate asset prices following rBergomi dynamics. Then, in Section \ref{sec:Details our approach and error bounds}, we explain the different building blocks that constitutes our proposed method, which are basically MISC solver, Brownian bridge construction and Richardson extrapolation. Finally, we show in Section \ref{sec:Numerical tests}, the results obtained through different numerical experiments, across different parameters constellations for the rBergomi model. Assuming one targets European call option prices estimates with a sufficiently small degree of error tolerance, the reported results   illustrate the substantial computational gains of our novel hierarchical method over standard MC, under different settings. 


