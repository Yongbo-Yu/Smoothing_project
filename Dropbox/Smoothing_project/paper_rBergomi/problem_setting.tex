In this section, we introduce the pricing framework that we are considering in this project. We start in Section \ref{sec:The rBergomi model} by giving some details for the rBergomi model proposed in \cite{bayer2016pricing}. We then derive the formula of the price of a European call option under the rBergomi model, in Section \ref{sec:Option pricing under rBergomi model}. This Section corresponds basically to the first stage of our approach, that is the analytical smoothing step. Finally, we explain in Section \ref{sec:Simulation of the rBergomi model}, some details about the hybrid scheme that we used to simulate the dynamics of asset prices under the rBergomi model.

\subsection{The rBergomi model}\label{sec:The rBergomi model}

We use  the rBergomi model for the price process $S_t$ as defined in  \cite{bayer2016pricing}, normalized to $r=0$, which is defined by

\begin{align}\label{eq:rBergomi_model1}
	dS_t = \sqrt{v_t(\widetilde{W}^H)} S_t dZ_t, \nonumber \\
	v_t = \xi_0(t) \exp\left( \eta \widetilde{W}_t^H - \frac{1}{2} \eta^2 t^{2H} \right),
\end{align}
where $0 < H < 1$ (Hurst parameter) and  $\eta>0$. We refer to $v_t$ as the variance process, and $\xi_0(t) = \expt{v_t}$ is  the forward variance curve.  Here, $\widetilde{W}^H $ is a certain Riemann-Liouville fBm
process,  defined by
\begin{align}\label{eq:Volterra process}
	\widetilde{W}_t^H = \int_0^t K^H(t,s) dW_s^1, \quad t \ge 0 \COMMA
\end{align}
where the kernel $K^H : \rset_+ \times \rset_+ \rightarrow \rset_+$ is
\begin{align*}
 \quad K^H(t,s) = \sqrt{2H} (t-s)^{H - 1/2},\quad \forall \: 0 \le s \le t.
\end{align*}

%We note that the map $s \rightarrow K^H(s,t)$ belongs to $L^2$, so
%that the stochastic integral \eqref{eq:Volterra process} is well defined.
 $\widetilde{W}^H $ is a centered, locally $(H-\epsilon)$- H\"older continuous, Gaussian process with $\text{Var}\left[\widetilde{W}^H_t \right] = t^{2H}$.

$W^1, Z$ denote two \emph{correlated} standard Brownian motions with correlation $\rho \in [-1,1]$, so that we can write
\begin{align*}
	Z:=\rho	W^1+ \bar{\rho}W^\perp \equiv \rho W^1+\sqrt{1-\rho^2} W^\perp,
\end{align*}
where $(W^1,W^\perp)$ are two independent standard Brownian motions.
Therefore, the solution to \eqref{eq:rBergomi_model1}, with $S(0)=S_0$, can be written as 

\begin{align}\label{eq:rBergomi_model}
	S_t&= S_0  \operatorname{exp}\left( \int_{0}^{t} \sqrt{v(s)} dZ(s)- \frac{1}{2} \int_{0}^{t} v(s) ds   \right),\quad S_0>0 \nonumber\\
	v_u&=\xi_0(u) \operatorname{exp}\left( \eta \widetilde{W}_u^H- \frac{\eta^2}{2} u^{2H} \right), \quad \xi_0>0 \PERIOD
\end{align}


The filtration $(\mathcal{F}_t)_{t\ge 0}$ can here be taken as the one generated by the two-dimensional Brownian motion $(W^1,W^\perp)$ under the risk neutral measure $\mathbb{Q}$, resulting in  a filtered probability space $(\Omega,\mathcal{F}, \mathcal{F}_t,\mathbb{Q})$. The stock price process $S$ is clearly then a local
$(\mathcal{F}_t)_{t\ge 0}$-martingale and a supermartingale, therefore integrable.  We shall henceforth use the notation $\expt{.} = E^{\mathbb{Q}}\left[. \mid \mathcal{F}_0\right]$ unless we state otherwise.








\subsection{Option pricing under the rBergomi model}\label{sec:Option pricing under rBergomi model}

We are interested in pricing European call options under the rBergomi model. Assuming $S_0 = 1$, and using the conditioning argument on the $\sigma$-algebra generated by $W^1$ (an argument first used by \cite{romano1997contingent} in the context of Markovian stochastic volatility  models), we can  show that the call price is given by

\begin{align}\label{BS_formula_rbergomi}
	C_{RB}\left( T, K \right) &= \text{E}\left[ \left(S_T - K \right)^+ \right]  \nonumber\\
	&=\expt{\expt{(S_T-K)^+ \mid \sigma(W^1(t) ,t \le T)}}\nonumber \\
	&=\text{E}\left[C_{BS}\left( S_0 = \operatorname{exp}\left(\rho \int_0^T \sqrt{v_t} dW_t^1 - \frac{1}{2}
	\rho^2 \int_0^T v_t dt\right),\ k = K, \ T = 1, \ \sigma^2 = (1-\rho^2)
	\int_0^T v_t dt \right) \right],
\end{align}
where $C_{BS}(S_0,k,\sigma^2)$ denotes the Black-Scholes call price, for initial spot price $S_0$, strike price $k$ and volatility $\sigma^2$.

To show \eqref{BS_formula_rbergomi}, we use the orthogonal decomposition of $S_t$ into $S_{t}^1$ and $S_{t}^2$, where

\begin{align*}
	S_t^1:=\mathcal{E}\{ \rho \int_{0}^{t}  \sqrt{v_s} dW_s^1\}, \: S_t^2:= \mathcal{E}\{ \sqrt{1-\rho^2} \int_{0}^{t}  \sqrt{v_s} dW_s^\perp  \}	,
\end{align*}

and $\mathcal{E}(.)$ denotes the stochastic exponential; then we obtain by conditional log-normality
\begin{align*}
	\log S_t \mid \mathcal{F}_t^1 \sim \mathcal{N}\left( \log S_t^1-\frac{1}{2} (1-\rho^2) \int_{0}^{t} v_s ds , (1-\rho^2) \int_{0}^{t} v_s ds \right),
\end{align*} 

where $\mathcal{F}_t^1= \sigma\{ W_s^1: s\le t\}$. Therefore, we obtain \eqref{BS_formula_rbergomi}.



We point out that the analytical smoothing, based on conditioning, performed in \eqref{BS_formula_rbergomi} enables us to get a smooth, analytic integrand inside the expectation. Therefore, applying sparse quadrature techniques becomes an adequate option for computing the call price as we will investigate later.

\subsection{Simulation of the rBergomi model}\label{sec:Simulation of the rBergomi model}

One of the numerical challenges encountered in the simulation of the rBergomi dynamics  is the computation of the terms  $\int_{0}^{T} \sqrt{v_t} dW_t^1$ and $V=\int_{0}^{T} v_t dt$ as in \eqref{BS_formula_rbergomi}, mainly because of the singularity of the Volterra kernel $K^H(s,t)$ at the diagonal $s = t$. In fact,  one needs to jointly simulate the two-dimensional Gaussian process $(W_t^1, \widetilde{W}^H_t: 0 \le t \le T)$, resulting in $W^1_{t_1},\dots, W_{t_N}$ and $\widetilde{W}^H_{t_1},\dots, \widetilde{W}^H_{t_N}$ along a given time grid $t_1 <\dots < t_N$. In the literature, there are essentially three possible ways to achieve this:
 \begin{enumerate}
 	\item[i)] Euler discretization of the integral \eqref{eq:Volterra process}, defining $\widetilde{W}^H$, together with classical simulation of increments of $W^1$. This is inefficient because the integral is singular and adaptivity probably does not help, as the singularity moves with time. For this method, we need an $N$-dimensional random Gaussian input vector to produce one (approximate, inaccurate) sample of $W^1_{t_1},\dots, W^1_{t_N}, \widetilde{W}^H_{t_1},\dots, \widetilde{W}_{t_N}$.
 	
 	\item[ii)] Given that $W^1_{t_1},\dots, W^1_{t_N}, \widetilde{W}^H_{t_1},\dots, \widetilde{W}_{t_N}$ together forms a ($2N$)-dimensional Gaussian random vector with computable covariance matrix. One can use Cholesky decomposition of the covariance matrix to produce exact samples of $W^1_{t_1},\dots, W^1_{t_N}, \widetilde{W}^H_{t_1},\dots, \widetilde{W}_{t_N}$, but unlike the first way, we need $2 \times N$-dimensional Gaussian random vectors as
 	input. This method is exact but slow (See  \cite{bayer2016pricing} and Section $4$ in \cite{bayer2017short} for more details about this scheme).   The simulation  requires $\Ordo{N^3}$ flops. 
 	
 	\item[iii)]  The hybrid scheme of \cite{bennedsen2017hybrid} uses a different approach, which is essentially based on  Euler discretization as the first way but crucially improved by moment
 	matching for the singular term in the left point rule. It is also
 	inexact in the sense that samples produced here do not exactly have the distribution of $W^1_{t_1},\dots, W^1_{t_N}, \widetilde{W}^H_{t_1},\dots, \widetilde{W}_{t_N}$, however they are much more accurate then samples produced from method $(i)$, but much faster than method $(ii)$. As in method $(ii)$, in this case, we need a $2 \times N$-dimensional Gaussian random input vector to produce one
 	sample of $W^1_{t_1},\dots, W^1_{t_N}, \widetilde{W}^H_{t_1},\dots, \widetilde{W}_{t_N}$.
 \end{enumerate}
In this project, we adopt approach  $(iii)$ for the simulation of the rBergomi asset price. We utilize the first order variant $(\kappa=1) $ of the hybrid scheme \cite{bennedsen2017hybrid}, which is based on the approximation

\begin{equation*}\label{eq:Hybrid_scheme}
\widetilde{W}^H_{\frac{i}{N}} \approx \bar{W}_{\frac{i}{N}}:= \sqrt{2H} \left(\int_{\frac{i-1}{N}}^{\frac{i}{N}} \left(\frac{i}{N} -s\right)^{H-\frac{1}{2}} dW_u^1+\sum_{k=2}^{i} \left(\frac{b_k}{N}\right)^{H-\frac{1}{2}} \left(W_{\frac{i-(k-1)}{N}}^1-W_{\frac{i-k}{N}}^1\right)\right) \COMMA
\end{equation*}
where $N$ is the number of time steps and 
$$ b_k:=\left(\frac{k^{H+\frac{1}{2}}-(k-1)^{H+\frac{1}{2} }}{H+\frac{1}{2}}\right)^{\frac{1}{H-\frac{1}{2}}} \PERIOD$$

Employing the fast Fourier transform to evaluate the sum in \eqref{eq:Hybrid_scheme}, which is a discrete convolution, a skeleton $\bar{W}_0^{H},\bar{W}_1^{H},\dots,\bar{W}_{\frac{[Nt]}{N}}^{H}$ can be generated in $\Ordo{N \log N}$ floating point operations.



The variates $\bar{W}_0^{H},\bar{W}_1^{H},\dots,\bar{W}_{\frac{[Nt]}{N}}^{H}$ are  generated by sampling $[nt]$ i.i.d draws from a $(\kappa+1)$-dimensional Gaussian distribution and computing a discrete convolution. We denote these pairs  of Gaussian random variables from now on by $(W^1,W^2)$.
