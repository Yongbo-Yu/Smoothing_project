
The rough Bergomi model, introduced recently in  \cite{bayer2016pricing}, is a promising rough volatility model in quantitative finance. This new model exhibits consistent results with the empirical fact of implied volatility surfaces being essentially time-invariant, and  ability to capture the term structure of skew observed in equity markets. In the absence of analytical European option pricing methods for the model, and due to the non-Markovian nature of the fractional driver, the prevalent option is to use Monte Carlo (MC) simulation for pricing. Despite the recent advances in MC method in this context, pricing under the rBergomi model is still a time consuming task. To overcome this issue, we design a novel,  alternative, hierarchical approach, based on adaptive sparse grids quadrature, specifically  multi-index stochastic collocation (MISC) as in  \cite{haji2016multi}, coupled with Brownian bridge construction and Richardson extrapolation. By uncovering the available regularity,  our hierarchical method demonstrates substantial computational gains with respect to the standard Monte Carlo method, assuming a sufficiently small error tolerance in the price estimates, across different parameter constellations, even for very small values of the Hurst  parameter.

\

\textbf{Keywords} Rough volatility, Monte Carlo, multi-index stochastic collocation, Brownian bridge construction, Richardson extrapolation.

\textbf{2010 Mathematics Subject Classification} 	91G60, 	91G20, 65C05, 65D30, 65D32.