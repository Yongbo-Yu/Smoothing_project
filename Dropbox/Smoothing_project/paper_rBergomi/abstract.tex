One of the recent rough volatility models that showed a promising potential in quantitative finance   is the rough Bergomi model, introduced by \cite{bayer2016pricing}. This new model exhibited consistent results with the stylised fact of implied volatility surfaces being essentially time-invariant, and  ablility to capture the term structure of skew observed in equity markets. In the absence of analytical European option pricing methods for the model,  the only prevalent option is to use Monte Carlo (MC) simulation for efficient pricing. We desing a novel alternative hierarchical approach based on adaptive sparse grids quadrature, specifically  multi-index stochastic collocation (MISC) as in  \cite{haji2016multi}, coupled with Brownian bridge construction and Richardson extrapolation.  Assuming one targets prices estimates with a sufficiently small degree of error tolerance, our new designed hierarchical  method  demonstrates substantial computational gains with respect to the standard Monte Carlo method, across different parameters constellations.

\

\textbf{Keywords} Rough volatility, Monte Carlo, multi-index stochastic collocation, Brownian bridge construction, Richardson extrapolation.

\textbf{2010 Mathematics Subject Classification} 	91G60, 	91G20, 65C05, 65D30, 65D32.