In this work,  we propose a novel hierarchical fast option pricer,  based on a  hierarchical adaptive sparse grids quadrature, specifically  multi-index stochastic collocation (MISC) as in  \cite{haji2016multi}, coupled with Brownian bridge construction and Richardson extrapolation, for options whose underlyings  follow rBergomi model as in \cite{bayer2016pricing}.  In order to use a  variant of adaptive sparse grids quadrature for our purposes, we had to solve two main issues, that constitutes the two stages of our new designed method. In the first stage, we smoothen the integrand by using the conditional expectation trick as was proposed by \cite{romano1997contingent}, in the context of Markovian stochastic volatility  models.   In a second stage, we applied a variant of hierarchical adaptive sparse grids quadrature, specifically MISC as in \cite{haji2016multi}, to solve the integration problem. In this stage, we had to apply two pre-transformations before using the MISC solver, in order to overcome the issue of facing a high dimensional integrand, due to the discretization scheme used for simulating the rBergomi dynamics. Given that MISC profits from anisotropy, the first pre-transformation consists of applying a hierarchical  path generation method, based on Brownian
bridge (BB) construction, with the aim of reducing the effective dimension. The second pre-transformation consists of applying Richardson extrapolation to reduce the bias, resulting in reducing the needed number of time steps in the coarsest level to achieve a certain error tolerance, and therefore,  the maximum number of dimensions needed for the integration problem.

Given that the only prevalent option, in this context, is to use different variants of MC method, our first contribution  is that we design an alternative approach based on  adaptive sparse grid quadrature, which opens a new persepective for experts in this field to investigate the performance of other methods besides MC, to solve pricing and calibration problems related to rBergomi model. Our second contribution is that  we reduce the computational cost not only through variance reduction, as in  \cite{mccrickerd2017turbocharging}, by using the conditional expectation, but also through bias reduction through using Richardson extrapolation, which contibutes also to reducing the dimension of the integration problem associated to computing the option price. Finally, assuming one targets prices estimates with a suffciently small degree of error tolerance, our thrid contribution is manifested by the observed substantial computational gains  over standard MC method, when pricing under rBergomi model. We show  these gains through our numerical experiments for  different parameters constellations. 

In this paper, we limit ourselves to compare our novel proposed method against standard MC. A more systematic comparison against the variant of MC proposed in \cite{mccrickerd2017turbocharging}  can be carried over but this is left for a future work. Another  eventual direction of research may also investigate the performance of QMC for such problems. Finally, accelerating  our novel designed method can be reached  by coupling MISC with a more optimal hierarchical path generation method than Brownian bridge construction, such as PCA or LT transformations, etc \dots