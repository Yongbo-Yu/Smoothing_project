In this work,  we propose a novel fast option pricer,  based on a  hierarchical adaptive sparse grids quadrature, specifically  MISC as in  \cite{haji2016multi}, coupled with Brownian bridge construction and Richardson extrapolation, for options whose underlyings  follow the rBergomi model as in \cite{bayer2016pricing}. 


% In order to use  adaptive sparse grids quadrature for our purposes, we solve two main issues, that constitutes the two stages of our new designed method. In the first stage, we smoothen the integrand by using the conditional expectation as was proposed by \cite{romano1997contingent}, in the context of Markovian stochastic volatility  models.   In a second stage, we apply  MISC as in \cite{haji2016multi}, to solve the integration problem. In this stage, we apply two transformations before using the MISC solver, in order to overcome the issue of facing a high dimensional integrand, due to the discretization scheme used for simulating the rBergomi dynamics. Given that MISC profits from anisotropy, the first transformation consists of applying a hierarchical  path generation method, based on Brownian
%bridge construction, with the aim of reducing the effective dimension. The second transformation consists of applying Richardson extrapolation to reduce the bias, resulting in reducing the needed number of time steps in the coarsest level to achieve a certain error tolerance, and therefore,  the maximum number of dimensions needed for the integration problem.

Given that the only prevalent option, in this context, is to use different variants of MC method, which is computationally expensive, our first contribution  is that we uncover the available regularity in the rBergomi model and  design a novel alternative approach based on  adaptive sparse grid quadrature, which opens a new research direction in this field to investigate the performance of other methods besides MC, to solve pricing and calibration problems related to the rBergomi model. Our second contribution is that we reduce the computational cost, through variance reduction,  by using the conditional expectation as in \cite{mccrickerd2017turbocharging} but also through bias reduction through using Richardson extrapolation. Finally, assuming one targets prices estimates with a sufficiently small error tolerance, our proposed method demonstrates substantial computational gains  over standard MC method, when pricing under the rBergomi model, even for very small values of the Hurst parameter. We show  these gains through our numerical experiments for  different parameters constellations.  \red{We clarify that we do not claim that these gains will hold, in the asymptotic regime, that is for higher accuracy requirements. Furthermore, the use of Richardson extrapolation is justified in the pre-asymptotic regime, in which our observed experimental results  show, in particular, an order one of convergence for the weak error}.


In this work, we limit ourselves to compare our novel proposed method against standard MC. A more systematic comparison against the variant of MC proposed in \cite{mccrickerd2017turbocharging}  can be carried out but this is left for a future work. Another  eventual direction of research may also investigate the performance of QMC for such problems. Finally, accelerating  our novel designed method can be reached  by coupling MISC with a more optimal hierarchical path generation method than Brownian bridge construction, such as PCA or LT transformations.