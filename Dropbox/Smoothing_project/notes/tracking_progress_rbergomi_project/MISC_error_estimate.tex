\subsubsection{MISC error estimate}


In our case, we have

\begin{align}\label{BS_formula_rbergomi_2}
C_{\text{RB}}\left( T, K \right)&=\text{E}\left[C_{\text{BS}}\left( S_0 = \operatorname{exp}\left(\rho \int_0^T \sqrt{v_t} dW_t^1 - \frac{1}{2}
\rho^2 \int_0^T v_t dt\right),\ k = K, \ \sigma^2 = (1-\rho^2)
\int_0^T v_t dt \right) \right] \nonumber \\
&\approx \int_{\rset^{2N}} C_{BS} \left(G(\mathbf{W}^{(1)},\mathbf{W}^{(2)})\right) \rho_{N}(\mathbf{W}^{(1)})  \rho_{N}(\mathbf{W}^{(2)}) d\mathbf{W}^{(1)} d\mathbf{W}^{(2)} \nonumber \\
&=C_{RB}^{N},
\end{align}
where $G$  maps  $2N$ independent standard Gaussian random inputs to the parameters fed to Black-Scholes formula, and  $\rho_N$ is the multivariate Gaussian density, given by 
\begin{equation*}\label{eq: multivariate gaussian distribution}
\rho_N(\mathbf{z})=\frac{1}{(2 \pi)^{N/2}} e^{-\frac{1}{2} \mathbf{z}^T \mathbf{z}} \PERIOD
\end{equation*} 
From  \eqref{BS_formula_rbergomi_2}, we define
\begin{equation*}
F^N=C_{\text{BS}}(G(\mathbf{W}^{(1)},\mathbf{W}^{(2)})) \COMMA
\end{equation*}
and introduce the set $C^0(\rset)$ of real-valued continuous functions over $\rset$, and the subspace of polynomials of degree at most $q$ over $\rset$, $\mathbb{P}^q(\rset) \subset C^0(\rset)$. Next,
we consider a sequence of univariate Lagrangian interpolant operators in each dimension $Y_n$ ($1 \le n \le 2N$), that is, $\{U_n^{m(\beta_n)}\}_{\beta_n \in \nset_+}$ (we refer to the value $\beta_n$ as the interpolation level). Each interpolant is built over a set of $m(\beta_n)$ collocation points, $\mathcal{H}^{m(\beta_n)}=\{y^1_n,y^2_n,\dots,y^{m(\beta_n)}_n\} \subset \rset$, thus, the interpolant yields a polynomial approximation,
\begin{equation*}
U^{m(\beta_n)}:C^0(\rset) \rightarrow \mathbb{P}^{m(\beta_n)-1}(\rset), \quad U^{m(\beta_n)}[F^N](y_n)= \sum_{j=1}^{m(\beta_n)} \left( f(y^j_n) \prod_{k=1;k \neq j}^{m(\beta_n)} \frac{y_n-y_n^k}{y_n^j-y_n^k}\right) \PERIOD
\end{equation*}
The $2N$-variate Lagrangian interpolant can then be built by a tensorization of univariate interpolants: denote by $C^0(\rset^{2N})$ the space of real-valued $2N$-variate continuous functions over $\rset^{2N}$ and by $\mathbb{P}^{\mathbf{q}}(\rset^{2N}) = \otimes_{n=1}^{2N} \mathbb{P}^{\mathbf{q}_n}(\rset)$ the subspace of polynomials of degree at most $q_n$ over $\rset$, with $\mathbf{q}=(q_1,\dots,q_{2N})\in  \nset^{2N}$, and consider a multi-index $\boldsymbol{\beta} \in \nset^{2N}_+$ assigning the interpolation level in each direction, $y_n$, then  the multivariate interpolant can then be written as
$$U^{m(\boldsymbol{\beta})}: C^0(\rset^{2N}) \rightarrow \mathbb{P}^{m(\boldsymbol{\beta})-1}(\rset^{2N}) ,\quad  U^{m(\boldsymbol{\beta})}[F^N](\mathbf{y})= \bigotimes_{n = 1}^{2N} U^{m(\beta_n)} [F^N](\mathbf{y}) \COMMA $$
Given this construction, we can define the MISC interpolant  for approximating $F^N$, using a set of multi indices $\mathcal{I} \in \nset^{2N}$ as
\begin{equation}
I^{\mathcal{I}}[F^N]= \sum_{\boldsymbol{\beta} \in \mathcal{I}} \Delta U_N^{\boldsymbol{\beta}} \COMMA
\end{equation}
where 
\begin{equation*}
\Delta_i U_N^{\boldsymbol{\beta}} = \left\{ 
\aligned 
 U_N^{\boldsymbol{\beta}} &- U_N^{\boldsymbol{\beta}'}  \text{, with } \boldsymbol{\beta}' =\boldsymbol{\beta} - e_i, \text{ if } \boldsymbol{\beta}_i>0 \\
 U_N^{\boldsymbol{\beta}} &, \quad  \text{ otherwise,}
\endaligned
\right.
\end{equation*}
where $e_i$ denotes the $i$th $2N$-dimensional unit vector. Then, $\Delta
U_N^{\boldsymbol{\beta}}$ is defined as
\begin{equation*}
\Delta U_N^{\boldsymbol{\beta}} = \left( \prod_{i=1}^{2N} \Delta_i \right) U_N^{\boldsymbol{\beta}}.
\end{equation*}
We define the interpolation error induced by MISC as
\begin{equation}
e_{N}= F^N-I^{\mathcal{I}}[F^N] \PERIOD
\end{equation}
One can have a bound on the interpolation error of MISC, $e_{N}$, by tensorizing one dimensional error estimates, and  then simply integrate that bound to get the MISC quadrature error, $\mathcal{E}_Q(\text{TOL}_{\text{MISC}},N)$. However, we think that this will  not lead to a sharp error estimate for MISC. Another strategy for estimating the MISC quadrature error, is to estimate $\expt{e_N}$ using MC by sampling directly $e_N$. 

If we define $Y=F^N+(Q_N^{\mathcal{I}}-I^{\mathcal{I}}[F^N])$ (where $Q_N^{\mathcal{I}}$ is the MISC quadrature estimator, then we have
\begin{align}\label{eq:Control_variate}
\expt{Y}&=\expt{F^N}\nonumber\\
Var[Y]&=Var[e_N]< Var[\mathcal{A}_{\text{MC}}]\COMMA
\end{align}
where $\mathcal{A}_{\text{MC}}$ is the MC estimator for $\expt{F^N}$.

\eqref{eq:Control_variate} shows that MISC can be seen as a control variate for MC estimator and consequently as a powerful variance reduction tool.


\textbf{TO-DO $1$}: Estimate numerically  $\expt{e_N}$ using MC by sampling directly $e_N$.

\textbf{TO-DO $2$}: Show numerically \eqref{eq:Control_variate}, that is ISC can be seen as a control variate for MC estimator. 

